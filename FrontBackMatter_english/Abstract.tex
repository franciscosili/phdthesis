% % Abstract

\thispagestyle{empty}
\pdfbookmark[0]{Abstract}{Abstract} % Bookmark name visible in a PDF viewer

\begin{center}
    %	\bigskip
    {\normalsize \href{https://unlp.edu.ar/}{\myUni} \\} % University name in capitals
    {\normalsize \myFaculty \\} % Faculty name
    {\normalsize \myDepartment \\} % Department name
    \bigskip\vspace*{.02\textheight}
    {\Large \textsc{Tesis de Doctorado}}\par
    \bigskip

    {\rule{\linewidth}{1pt}\\%[0.4cm]
    \Large \myTitle \par} % Thesis title
    \rule{\linewidth}{1pt}\\[0.4cm]

    \bigskip
	{\normalsize by \myName \par} % Author name
    \bigskip\vspace*{.06\textheight}
\end{center}


{\centering\Huge\textsc{\textbf{Abstract}} \par}
\bigskip

some text

% This thesis presents a search for the production of supersymmetric gauge bosons decaying via scalar tau leptons into a final state of at least two same-sign hadronically decaying tau leptons.  This search has been performed analysing 13 TeV proton-proton collision data recorded by the ATLAS detector at the CERN Large-Hadron-Collider during the years of 2015-2018.  A total of 139 fb$^{-1}$ of data has been analysed.  No significant excesses have been observed with respect to the Standard Model expectation,  therefore exclusion limits have been set at 95\% confidence level.  Masses of the lightest chargino (\Cone), equivalent with the mass of the next-to-lightest neutralino (\Ntwo), up to 960 GeV for massless lightest neutralinos \nolinebreak (\None\nolinebreak) \nolinebreak have been excluded.  Mass differences between the \Cone and \None as little as 30 GeV have been excluded for a \Cone mass of 80 GeV.  A statistical combination with an ATLAS analysis using an opposite di-tau final state has been performed,  leading to an exclusion of \Cone / \Ntwo masses up to 1160 GeV for massless \None~.  
% Dedicated studies on the performance of electron triggers are also presented within this thesis that have been  part of the author's qualification task and continued involvement within the ATLAS trigger community.  These studies have hinted towards the need of dedicated electron trigger correction factors in the ATLAS fast simulation framework and the author has provided a first set of such scale factors to the ATLAS Collaboration. 


\noindent 
