\chapter{Data Acquisition,  Reconstruction and Monte Carlo Simulation}
\label{ch:DAQ}
\epigraph{\emph{“Champions keep playing until they get it right.”}}{Billie Jean King}

yet another template (yat)

% %\section{Data Aquisition}
% A crucial step in the recording of collision events with ATLAS is the selection through dedicated signature triggers.  These trigger decisions are emulated in Monte Carlo simulation in order to replicate the selection in data.  
% After events have been selected by the ATLAS trigger system and saved to disk,  further detailed reconstruction and identification algorithms can be performed.  In the following,  a short overview of the signature triggers of importance to this thesis as well as a summary of the reconstruction and identification processes for the particles and physics objects used in this thesis is given. 
% Additionally,  a brief description of Monte Carlo concepts and ATLAS detector simulation is given.

% \section{Signature triggers}
% \label{sec:DAQ:SignatureTriggers}

% %have to include a discussion about prescaled triggers, unprescaled triggers as well as RERUN triggers! also be clear what a trigger chain is! 
% %need to talk about trigger 'budget' or how the rate reduces from step to step,  e.g. referring to trigger budget of 1kHz at the HLT  in the trigger chapter
% %have to mention isolation in the electron trigger chain case
% An introduction to the ATLAS trigger system was given in section \ref{sec:ExpSetup:trigger}; in the following an overview of the trigger signatures of importance to this thesis are discussed.
% All the trigger signatures combined form a \textit{trigger menu},  a selection of triggers  simultaneously active in ATLAS in order to select all events of interest to the multitude of physics analyses in the ATLAS collaboration. 
% A trigger is a unique identifier for a sequence of trigger algorithms and the requirements that have to be fulfilled after each algorithm step,  including a specification of the Level-1 hardware trigger requirements.  A signature here refers to the combination of specific interactions that different \ac{SM} particles have with the various ATLAS detector layers.  

% %\begin{figure}[htpb!]
% %\centering
% %\includegraphics[width=0.6\linewidth]{figures/DAQ/TrigOpsPublicWinter2019_HLT_group_rates_ATLASStyle_359872.pdf}
% %\caption{Exemplary overview of physics stream trigger rates at the \ac{HLT} as a function of time.  \label{fig:DAQ:TriggerBudget}}
% %\end{figure}

% %Taken from \textit{https://twiki.cern.ch/twiki/bin/view/AtlasPublicTriggerOperationPublicResults#Trigger_rates_and_bandwidth}

% \subsection{Electron triggers}
% The electron trigger signature uses signals from the electromagnetic and hadronic calorimeters at Level-1 to select high energetic clusters of calorimeter towers.  Based on that decision,  a fast selection,  followed by precision algorithm-based selections at HLT are performed.  Very detailed work on the electron trigger has been part of the PhD work of the candidate,  therefore  a detailed discussion of the electron trigger signature as well as its performance in Run-2 is discussed in chapter \ref{ch:trigger}.
 
% \subsection{Muon triggers}
% The muon trigger signature relies on signals from the \ac{MS} and tracks from the \ac{ID}.  
% In the Barrel of the detector,  the Level-1 muon trigger signal consists of coincidences of hits in two or three \ac{RPC} segments for high and low $p_T$ momentum muons,  respectively.  
% In the End-Cap region ($1.04 < \abs{\eta} < 2.4$),  the  Level-1 muon decision is based on coincidences in the \ac{TGC} segments of the Big Muon Wheel.  After this first Level-1 step,  information on \ac{ID} tracks is accessible at \ac{HLT}.  
% Multiple isolation requirements on the muons at trigger level can be applied. 
% The lowest unprescaled single muon trigger uses a lower momentum threshold of 20 GeV in 2015,  raised to 26 GeV from 2016 onwards.  The efficiency of the lowest unprescaled single muon trigger between 2016 and 2018 is shown in Figure \ref{fig:DAQ:muontrigger}.  
% An extensive description of the muon trigger and its performance in Run-2 can be found in \cite{MuonTriggerPerformance},  here only a short overview can be given.
% The efficiency of the trigger is calculated on recorded collision events with selections targeting events with Z-bosons decaying into a opposite-sign di-muon final state.  A selection of muons from this decay,  is used as unbiased \textit{probe} sample to test the trigger's selection efficiency.  The trigger efficiency is defined as a ratio of the selected probe muons fullfilling the trigger selections over all available probe muons. The trigger in Figure  \ref{fig:DAQ:muontrigger} shows a fast turn-on of its efficiency in dependency of the probe muon's momentum,  reaching almost 70\% efficiency in the barrel regions and 90\% in the detector end-caps,  with a close agreement between data and Monte Carlo simulation visible in the lower panels.   
% \begin{figure}[h]
% \centering
% \includegraphics[width=0.95\linewidth]{figures/DAQ/MuonTrigger.png}
% \caption{Efficiency of the lowest unprescaled muon trigger in data collected between 2016 and 2018 in the detector barrel ($|\eta| < 1.05$) and end-cap ($1.05 < |/eta| < 2.5$) regions\label{fig:DAQ:muontrigger}, from \cite{MuonTriggerPerformance}}
% \end{figure}
% The lower trigger efficiency observed in the barrel region compared to the end-cap is due to a smaller geometric coverage at the Level-1 trigger stage.  

%  \subsection{Missing transverse momentum triggers}

% Missing transverse energy (\Met) in the detector is one of the essential features in many \ac{BSM} searches in ATLAS.  In connection with R-parity conserving \ac{SUSY},  the lightest neutralino particle is stable and can escape the detector without being detected. 
% This detector signature is similar to neutrinos escaping the detector. 
% Triggering on missing transverse momentum in this thesis is used in combined triggers with taus.
% The \MET  trigger is solely based on calorimeter information; multiple algorithms are available on the High-Level-Trigger to calculate the missing transverse momentum,  varying in the definition of used calorimeter clusters and their calibration.
% In essence,  a projection and summation of the clusters as described in equation \eqref{eqn:daq:met} \cite{MetTrigger} is performed.  
% \begin{align}
% \begin{split}
% E_x^\text{miss} = - \sum_{i=1}^{|\text{Elements}|} E_i \sin\theta_i \cos\phi_i\\
% E_y^\text{miss} = - \sum_{i=1}^{|\text{Elements}|} E_i \sin\theta_i \sin\phi_i
% \label{eqn:daq:met}
% \end{split}
% \end{align}
% A detailed explanation of the \MET trigger and its performance in Run-2 can be found in \cite{MetTrigger}.  For data analysis and selection discussed in this thesis,  only the most rudimentary missing transverse energy \ac{HLT} algorithm is used.  This is extracting \Met from a sum of all calorimeter cells with their uncalibrated energy above a noise threshold.  This noise threshold is included to avoid electronic noise and pile-up contributions.

%   \subsection{Tau Triggers}
%   \label{sec:daq:tautriggers}
% The tau trigger signature is based on a fast selection of electromagnetic and hadronic clusters at Level-1,  followed by reconstruction and identification algorithms run at the \ac{HLT}.  A detailed description can be found in \cite{TauTrigger}; a summary is given below.\\
% The applied selections are based on hadronic tau decays happening before traversing the active detector region. These signatures are showing as narrow calorimeter energy deposits with a small number of associated tracks. 
% At ~Level-1, ~ a core region made of $2\times2$ trigger towers ( each $0.1 \times 0.1$ in $\Delta\phi \times \Delta \eta$) is defined, with the $E_T$ of the hadronic tau candidate made up of the two most energetic consecutive cells. 
% Around this core region, an isolation region of $4\times4$ trigger towers is used to determine the isolation in the electromagnetic calorimeter. 

% The tau energy reconstruction at Level-1 does not include any calibrations or clustering algorithms and has a worse energy resolution than observed in the offline tau reconstruction. This leads to a signal efficiency loss for low transverse energy tau candidates.
% Regions-of-Interest selected by the Level-1 are passed to the HLT. 
% The tau selection here includes three steps: a calorimeter only selection, a track preselection, and an offline-like selection. 
% The calorimeter-only selection step uses the full detector granularity,  making use of the topological clustering algorithm \cite{TopoClusters}. This algorithm,  generating so-called \textit{Topo-Clusters} considers the significance of the electronic signal of each calorimeter cell and removes calorimeter cells if they are not close to cells with a large signal.  This topological clustering includes information about the shape and location of the cluster, allowing for calibrations to be applied.  In the tau trigger case described here,  a simplified tau energy calibration is applied. 
% The track preselection step makes use of a two-stage fast-tracking reconstruction. 
% At first, tracks within a narrow area along the beam axis are reconstructed, finding the tau vertex. In a second step, tracks in a larger $\Delta R$ window are considered, but in a narrow interval around the tau vertex. 
% This allows for fast reconstruction of the vertex and significantly reduces the CPU usage compared to a reconstruction of tracks in a large $\Delta R$ and z window.
% Eventually,  an offline-like selection is applied. 
% The tracks reconstructed in the previous step are now passed through precision tracking and therefore available with higher accuracy.  Together with the calibrated clusters,  multiple kinematic variables are used as input to a Boosted Decision Tree. This \ac{BDT} is similar to an identification method previously used to identify offline taus.  The \ac{BDT} is trained to distinguish taus from jets, using features such as the maximum distance in $\Delta R$ of the tracks associated to the tau,  with tracks associated with jets expected to be more wide-spread than tau associated tracks.  A detailed description of the \ac{BDT} can be found in \cite{TauRecoBDT}. %fix this reference: https://cds.cern.ch/record/2064383
% Multiple tau identification working points are defined based on fixed cuts on the \ac{BDT} output score.  At trigger level,  a medium working point is used. 
% %To allow for low tau pt thresholds, an additional requirement at level  1 must be made.  L1Topo is used to ask for an additional jet in the event. This is necessary to lower the trigger rates. 

 
%  \section{Object Reconstruction}
%  \label{sec:DAQ:ObjectReco}
% After the selection of events through the signature triggers described above,  the particle signatures are reconstructed and identified \textit{offline},  after they have been saved to storage.  This reconstruction is done event by event with the same procedure applied to Monte Carlo simulation and recorded ATLAS data. 
% Each particle has its reconstruction and identification procedure and can have additional algorithms to help with the specification of its properties.  Since the event is already recorded,  less limitations on the CPU consumption and timing constraints governing the online reconstruction can be used. 
% In the following,  a brief overview of the offline reconstruction and identification of the objects used in this thesis is given. 

% \subsection{Electrons}
%   \label{sec:DAQ:ObjectReco:Electrons}
 
% The main features of an electron or positron (in the following both referred to as electron) in ATLAS is a cluster of energy deposits in the \ac{ECAL},  tracks in the \ac{ID}  as well as a close match between the clusters and tracks in terms of $\eta \times \phi$.
% These three main features present the three consecutive steps in reconstructing electrons: first,  a seed cluster is reconstructed: Based on a sliding-window algorithm,  looking at collections of energy towers with at least 2.5 GeV of summed transverse energy. 
% Here an energy tower is the sum of collected energy in all three layers of the \ac{ECAL},  including the presampler,  in a $\Delta \eta \times \Delta \phi = 0.025 \times 0.025$ segment. This mimics the granularity of the second layer of the \ac{ECAL}.
% In Figure \ref{fig:DAQ:egammareco},  the path of an electron through the different detector components is illustrated,  also highlighting the three different layers in the \ac{ECAL}.

% \begin{figure}
% \centering
% \includegraphics[width=0.8\linewidth]{figures/DAQ/EgammaReco.png}
% \caption{Illustration of an electron path through the Inner Detector and Calorimeters of ATLAS,  highlighting the layer granularity in the electromagnetic calorimeter.  The dashed line presents a photon produced through interactions of the electron with the \ac{ID} material. Taken from \cite{ElectronRecoID1516}. \label{fig:DAQ:egammareco}}
% \end{figure}

% After the seed cluster is identified,  tracks are reconstructed.  This is based on pattern recognition of hits in the \ac{SCT} and Pixels,  followed by an ambiguity resolution step,  resolving ambiguity of hits associated with multiple tracks, and an extension of the tracks to the \ac{TRT}.
% For the fitting of tracks,  a global $\chi^2$ fitting procedure is used,  with either a pion hypothesis or an electron hypothesis in the presence of significant Bremsstrahlung. 
% Consecutively,  a \ac{GSF} \cite{GSF} is applied to account for energy losses in the \ac{ID} material. 
% As the last electron reconstruction step,  a matching of the GSF-track and calorimeter cluster is performed.  This matching is based on a geometric distance between the clusters and tracks and is described in detail in \cite{ElectronRecoID1516}.  The electromagnetic energy clusters are further calibrated as described in \cite{ElectronCalib}.

% After the reconstruction,  a further identification is necessary to distinguish prompt electrons from electrons originating from misidentified hadrons, non-isolated electrons from heavy-flavour decays or electrons from photon conversion. 

% This identification is based on a likelihood discriminant construction.  The likelihood discriminant $d'_L$ is given in equation\eqref{eqn:daq:eleclikelihood} \cite{ElectronRecoID1516},  where a fixed factor $\tau = 15$ is used to achieve a smooth discriminant distribution.   

% \begin{align}
% \begin{split}
% d_L &= \frac{L_S}{L_S + L_B} \\
% d'_L &= -\tau^{-1} \ln(d^{-1}_L -1)
% \label{eqn:daq:eleclikelihood}
% \end{split}
% \end{align}

% Here $L_S$ describes the likelihood function of signal,  prompt electrons. Whereas $L_B$ is the likelihood of non-promt,  fake electrons.  The likelihoods $L_S$ and $L_B$ are a product of probability density functions for a large set of kinematic variables.  The distributions and the likelihoods are binned in the $p_T$ and $\eta$ of the electron candidate.  A complete set of the variables considered in the likelihood definition is included in \cite{ElectronRecoID1516}, as well as a detailed description of the pdf and likelihood extraction.  
% Using a multi-variate technique can help where a simple cut on single variables would not offer strong discriminating power.  
% %based on discussion and drafting with Stefan Richter! 
% Based on the output of the discriminant,  several working points are defined.  These are designed to have a given signal-electron selection efficiency in each $(p_T, \eta)$ bin. Therefore a cut on the discriminant is defined in each $(p_T, \eta)$ bin.  In addition, to stabilise the performance at high pileup (to avoid getting too much increase in background passing the selection),  the cut on the discriminant is done such that it depends linearly on a pileup variable.  
% %in each $p_T-\eta$ bin i
% Effectively,  the cut gets tightened as pileup increases.  The number of primary vertices in the event is used as the pileup variable for offline electrons.  For online electrons, this is too computationally intense, so the number of inelastic collisions per bunch crossing ($\mu$) is used instead as a measure of the amount of pileup.
% Additional to the electron likelihood based identification,  isolation requirements can be applied to distinguish between prompt electrons and electrons from heavy-flavour decays.  Two isolation criteria for offline electrons are defined,  \textit{FCTight} and \textit{FCLoose}.  These isolation working points include a restriction on the transverse energy sum in a $\Delta R <0.2$ region around the electron in the calorimeter ($E_T^\text{iso}$) to be below 0.06 and 0.2 times the electrons energy ($E_T$), for \textit{FCTight} and \textit{FCLoose}, respectively.  The calorimeter based isolation requirement is combined with a track-based isolation, restricting the momentum.  The radial distance has a maximum value of 0.2 and decreases with the electrons momentum.  For the \textit{FCTight} working point,  the momentum fraction within the isolation cone is restricted to be smaller than  0.06 times the electrons momentum,  for \textit{FCLoose} it is below 15\% of the electrons momentum. 
% %pileup distr in each pt eta bin - schnitt auf disk linear function of pileup dependency
% %id menu dep pt eta and pileup 
% %disc schnitt (Y) pileup (X) - steigung konstruiert 0 very loose , tight am stärksten - damit tight immer medium erfüllt und es nicht vorkommen kann dass tight für hohes pileup medium nicht erfüllt

  
%   \subsection{Muons}
% Muon candidates as minimum ionising particles in ATLAS are reconstructed using track segments from the Muon Spectrometer that have been matched to \ac{ID} tracks.  Energy loss in the calorimeters is taken into account in a combined fitting of \ac{MS} and \ac{ID} tracks.  
% Multiple identification working points in varying levels of prompt muon selection efficiency and background rejection are defined,  designed to satisfy the large varying needs of physics analyses.  Additional to identification criteria,  an isolation requirement can be applied \cite{MuonID}.  Similar to the electron isolation working points,  a muon isolation based on calorimeter and track-based isolation criteria is defined,  \textit{FCLoose}.  The transverse energy in a  $\Delta R < 0.2$ isolation cone around the muon is restricted to 30\% of the muons momentum.  Addtionally, the momentum in a variable size, maximum $\Delta R < 0.3$ cone around the muon should not extend 15\% of its momentum. 

% \subsection{Jets}
% The particle-flow jet reconstruction \cite{ParticleFlow} makes use of calibrated Topoclusters as well as tracking information from the \ac{ID}.  The topological clustering algorithm has been previously introduced and discussed in section \ref{sec:daq:tautriggers} in relation to tau triggers. The anti-$k_T$ sequential recombination algorithm \cite{antikt} is used to form jet objects based on particle-flow objects.  To remove pile-up jets,  a multivariate discriminant based on vertex information,  the \ac{JVT} \cite{JVT} can be used.

% %topological clusters of calorimeter cells 

% \subsection{Taus}
% \label{sec:DAQ:ObjectReco:Taus}
% As briefly highlighted within the description of the tau trigger in section \ref{sec:daq:tautriggers},  hadronic tau decays have a distinct detector signature that can be used to trigger on events containing hadronic taus.  In the following,  a brief discussion of the tau decay is given to provide a baseline of the features important to this thesis, before summarizing the hadronic tau reconstruction and identification within \ac{ATLAS}.
% Tau leptons have a mass of $1776.86  \text{ MeV}$ and a mean lifetime of $290.3 \times 10^{-15} \text{ s}$ \cite{PDG2022}. With their proper decay length around $87 \mu\text{m}$ \cite{PDG2022},  taus decay before interacting with active detector layers of \ac{ATLAS}.  As can be seen in Figure \ref{leptaudecay},  tau leptons decay via a $W$ boson and a tau neutrino.  The hadronic or leptonic decay of the tau is determined through the consecutive decay of the $W$-boson,  Figure \ref{leptaudecay}, highlighing a leptonic decay into an electron. Leptonic decays of the tau make up 35\% of tau decays, with hadronic decays making up 65\% \cite{PDG2022}.  Examples of hadronic tau decays are given in Figure \ref{hadtaudecay},  highlighting the calorimeter systems in which the decay products will leave the majority of their energy.  Not explicitely shown in the sketch is the $W$-boson,  only its decay products.  In 72\% the hadronic tau decay includes one charged pion,  in 22\% of the cases the decay includes three charged pions.  These charged pions leave tracks in the \ac{ID},  presenting a distinct identfication feature of taus,  the number of charged tracks or \textit{prongs} associated with their decay. 
% %t 65% of all possible decay modes [1]. In these, the hadronic decay products are one or three charged pions in 72% and 22% of
% %all cases, respectively. 
% The hadronic decay of a tau lepton is determined through the quarks and intermediate mesons built in the $W$-boson decay.  Examples thereof are shown in Figure \ref{hadtaudecay}. 
% \begin{figure}[h]
% \subfloat[\label{leptaudecay}]{\includegraphics[width=0.25\linewidth]{figures/tauLep.pdf}}
% \subfloat[\label{hadtaudecay}]{\includegraphics[width=0.7\linewidth]{figures/taudecay.pdf}}
% \caption{Feynman diagram of a leptonic tau decay,  highlighting its decay via a W-boson (a). Examples of hadronic tau decays into (from left to right) one charged pion,  one charged and one neutral pion as well as three charged pion (b)\cite{TauDecayTikz},  highlighting in which detector systems the participating particles will be detected.  \label{taudecay}}
% \end{figure}

% Hadronic tau candidates are seeded from jets that have been reconstructed using the anti-$k_T$ algorithm,  with a distance parameter of $\Delta R =0.4$  \cite{TauReconstruction}.  Additionally,  jet seeds are required to have $p_T > 10 \text{ GeV}$ and $|\eta| <2.5$.
% %A tau vertex is identified among all primary vertex candidates that lies within $\Delta R < 0.2$ of the jet axis.
% A \ac{BDT} is used to categorize all tracks associated to the tau candidate within $\Delta R = 0.4$ of the tau axis to be core or isolation tracks.  The number of core tracks is used to define the prongness of taus,  indicating the number of charged particles involved in the tau's decay.

% To distinguish tau particles from jets,  a recurrent neural network (RNN) identification is used.   
% For this,  multiple kinematic variables are used for the training of an RNN \cite{TauRNNID}.    
% Variables include specifications of the cluster depth, the longitudinal extension and the radial cluster extension as well as fractions of momenta in the core or isolation region of the tau cluster,  to highlight a few of them. 
% The performance of the RNN identifier is shown in Fig. \ref{fig:DAQ:tauPerformance} for its 1-prong and 3-prong decay modes. This is also compared with the performance of a BDT identifier previously used in the $\tau$ identification in ATLAS.  The improved network architecture of the RNN identifier provides  rejection of jets by up to a factor of two better than the previous BDT approach. 

% \begin{figure}[h]
% \centering
% \includegraphics[width=0.6\linewidth]{figures/DAQ/RNNPerformance.png}
% \caption{Performance of the tau \ac{RNN} identification working points in comparison with previously used \ac{BDT} based identification \label{fig:DAQ:tauPerformance} \cite{TauRNNID}}
% \end{figure}

% \subsection{Missing transverse energy}
% The reconstruction of missing transverse energy after an event has been stored by a trigger is similar to the procedure at trigger level described in section \ref{sec:DAQ:SignatureTriggers}.  In contrast to the trigger reconstruction,  the offline missing transverse energy reconstruction is not based on uncalibrated clusters, but fully calibrated physics objects.  Additionally to this part of the \Met reconstruction considering physics object,  a soft term comprised of low momentum tracks is taken into account \cite{MetPerformance}.

% \section{Monte Carlo Simulation}
% \label{sec:DAQ:MC}
% To evaluate any measurements taken with the ATLAS detector,  simulations of physics processes and their decay within the ATLAS detector are necessary.  This is achieved through the use of Monte Carlo event generators.  In the following,  a brief description of the essentials of Monte Carlo simulations are given,  based on pedagogic introductions given in \cite{MCPedagogic}.  Followed by a short overview of the Monte Carlo generators used in this thesis.
% For the production of a final state X through the collision of two hadrons ($h_1$, $h_2$),  the inclusive cross section can be factorised like the following:

% \begin{align}
% \sigma_{h_1,h_2\rightarrow X} = \sum_{a,b\in {q,g}} \int dx_a \int dx_b f_a^{h_1}(x_a,\mu_F^2)f_b^{h_2}(x_b, \mu_F^2) \int \,d\Phi_{ab\rightarrow X} \frac{d\hat{\sigma}_{ab}(\Phi_{ab\rightarrow X},\mu_F^2)}{\,d\Phi_{ab\rightarrow X} }
% \label{eq:DAQ:xsec}
% \end{align}
% With $f_a^{h_1}(x_a,\mu_F^2)f_b^{h_2}(x_b, \mu_F^2)$ representing the parton distribution functions. 
% Here $\mu_F$ presents the factorisation scale.  This scale is the cutoff between processes that are considered perturbatively and the non-perturbative regime.  
% %In order to evaluate the systematic uncertainty caused by the cross section's dependence on the factorisation scale,  this scale is varied in alternative samples.
% An important part of this inclusive cross section is presented by the Matrix element describing the hard scattering process,  this is included in the last integral of equation \eqref{eq:DAQ:xsec}.
% After the inclusive matrix element calculation as described in equation \eqref{eq:DAQ:xsec},  the parton showering takes into account the additional particles that can contribute to the production.  The parton showering here is considering that a parton can originate from a splitting of  other partons.
% To not consider the additional particles twice,  a matching of the matrix element and the parton showering has to be performed.  
% The produced partons in such a showering will form colourless hadrons,  which in turn can further decay.  A last aspect to be taken into consideration for event simulations is the underlying event. This is caused by \textit{spectator} partons, not directly participating in the hard scattering.


% \subsection{Monte Carlo generators}

% \paragraph{Sherpa}
% One of the general purpose event generators used in this thesis is \texttt{SHERPA} \cite{Sherpa}.  It includes matrix element generators as well as a built-in parton showering.  In this thesis,  simulated events with \texttt{SHERPA} include Multi-boson processes as well as the associate production of vector bosons and jets.  

% \paragraph{Powheg} 
% The \texttt{POWHEG} \cite{Powheg} framework is a matrix element generator at next-to-leading perturbative order.  The matrix element generation is interfaced with generators like \texttt{PYTHIA} or \texttt{HERWIG++} in order to simulate the parton shower.  This kind of combination of matrix element generator and parton shower simulation is used or top-related processes such as top-antitop pair production. 

% \paragraph{MadGraph5\_ aMC@NLO} \texttt{MadGraph5} \cite{Madgraph} is used as a matrix element calculator at next-to-leading perturbative order.  Similar to \texttt{POWHEG},  it is interfaced with \texttt{PYTHIA} or \texttt{HERWIG} to include parton showering. 

% \paragraph{Pythia}
% A second general purpose event generator used in this thesis is \texttt{PYTHIA 8}\cite{pythia},  even with matrix element calculation and parton showering possible within \texttt{PYTHIA}, it is widely used for its parton showering,  interfaced with \texttt{POWHEG} or \texttt{HERWIG}.  Additionally,  \texttt{PYTHIA} is used in this thesis to simulate minimum-bias proton-proton collisions.

% \paragraph{Herwig} The \texttt{HERWIG} \cite{Herwig,Herwig++} Monte Carlo event generator has capabilities to be used to simulate the matrix element,  but is only used within this thesis as a variation of the \texttt{PYTHIA} parton showering. 

% \paragraph{EvtGen}
% \texttt{EvtGen} \cite{EvtGen} is a framework that simulates the decay of final state particles.  Events associated with the production and decay of a top quark used in this thesis are simulated through a combination of \texttt{POWHEG}, \texttt{PYTHIA} and \texttt{EvtGen}.  Within this thesis, all events generated with a parton showering by \texttt{PYTHIA} include final state particle decays simulated with \texttt{EvtGen}.

% \subsection{ATLAS detector simulation}
% \label{sec:DAQ:AF2}


% To directly compare the data collected with the ATLAS detector with the prediction of \ac{SM} and \ac{BSM} events in simulation,  the interaction of the produced particles with the detector material has to be simulated.
% The Geant4 \cite{Geant4} software package is used to simulate the interaction of particles with the detector material.
% A full Geant4 model of the ATLAS detector is used to simulate the transition of particles produced in proton-proton collisions through the different detector layers. 

% The simulation of a large number of interactions necessary to mimick the ATLAS reconstruction is computationally extensive.  Especially the simulation of shower developments in the calorimeters consumes a large amount of CPU and computing time. 
% For many \ac{BSM} searches,  a large number of parameters affecting the predicted particle masses and interactions have to be simulated.  A 'fast' parameterised detector simulation has been developed to cope with this high simulation demand.  A so-called Atlfast-II or AFII setup simulation chain uses Geant4 simulation for the interactions in the \ac{ID} and muon spectrometer,  but a parametrised simulation called FastCaloSim for the particle interactions in the electromagnetic and hadronic calorimeter. 
% The improvements in computing time compared to Geant4 simulation of the full detector as well as a fast,  simplified Geant4 simulation is shown in Figure \ref{fig:DAQ:AFIIvsG4}.  The overall processing time is reduces by roughly an order of magnitude compared to the full Geant4 simulation \cite{AFIIprinciple}. 
% \begin{figure}[h]
% \centering
% \includegraphics[width=0.8\linewidth]{figures/DAQ/AFII_CPU_improvement.png}
% \caption{CPU time distributions for 250 $ \ttbar $ events compared for G4,  fast G4 and AFII setup,  taken from \cite{AFIIprinciple} \label{fig:DAQ:AFIIvsG4}}
% \end{figure}
% The fast calorimeter simulation FastCaloSim uses a parametrisation of the calorimeter response.  The parametrisation has been extracted through Geant4 simulation and tunes to data. 
% The main three simplifications include simplifying the detector geometry,  approximating the calorimeter cells as cuboids,  only reproducing the average lateral energy distributions and restricting the simulation to three types of initial particles. 

