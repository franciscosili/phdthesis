\FloatBarrier
To test the background-only hypothesis versus the signal plus background hypothesis, a complete statistical model in the form of a likelihood is used. This likelihood as given in equation \eqref{eq:stats:likelihoodWGaussian} is simultanesously fitted in all control regions to extract normalisation factors on the backgrounds defined through control regions. 
The result of the simultaneous likelihood fit for these normalisation factors can be seen in Table \ref{tab:C1N2SSNormFactors}, which includes a normalisation of the Multi-jet contribution extracted in CRA-LM,  a W+jets normalisation (W-CR) as well as a top background normalisation for the high mass case (TCR-HM). 
All values of normalisation factors described here are extracted simultaneously and therefore taking into account the effect of one normalised background onto another in each control region.  %The control regions are designed to keep the effects of other backgrounds minimal which is achieved by a high purity in the targeted background processes.

\begin{table}[h]\centering
\begin{tabular}{|l|c|} \hline
&Normalisation factor \\ \hline
%$\mu_{MJLow} $& $0.9992 \pm  0.0427$\\
%$\mu_W       $& $1.035 \pm 0.0883  $\\
%$\mu_{topHigh } $& $0.7073 \pm 0.105$\\ \hline
$\mu_{MJLow} $& $1.00 \pm  0.04$\\
$\mu_W       $& $1.04 \pm 0.09  $\\
$\mu_{topHigh } $& $0.71 \pm 0.11$\\ \hline
\end{tabular}
\caption{Normalisation factors for the Multijet contribution ($\mu_{MJLow}$) as well as $W$+jets and Top contribution \label{tab:C1N2SSNormFactors}}
\end{table}

Including CR-A of the ABCD multijet estimation for the low mass scenario (see Table \ref{tab:bkgestimation:QCD:table}) into the likelihood allows for a normalisation factor of the Multi-jet background. This is accounting for residual effects of the ABCD method (section \ref{sec:bkgestimation:ABCD}),  which is performed before the likelihood construction.
A comparison of a kinematic distribution in CR-A  of the low mass scenario before the background-only fit with the multijet VR-F (MJVR) is given in Figure \ref{fig:results:mjcrvr}. Clearly,  the agreement of data and \ac{SM} expectation in the control region is very good,  even pre-fit (see Fig. \ref{fig:results:mjcrprefit}).  This is confirmed in the validation region (see Fig. \ref{fig:results:mjvrpostfit}) post-fit.   

\begin{figure}[htpb!]
%\subfloat[width=0.3\linewidth]{figures/MJCRA_prefit.pdf}
\subfloat[CRA-LM pre-fit \label{fig:results:mjcrprefit}]{\includegraphics[width=0.49\linewidth]{figures/UpdatedThesisFigures/CRA_lowmass_prefit_noATLAS.pdf}}
\subfloat[MJ-VR post-fit \label{fig:results:mjvrpostfit}]{\includegraphics[width=0.49\linewidth]{figures/PaperDraft/fig_abcd-C1N2SSL.pdf}}
\caption{pre-fit Multijet CR-A (as defined in the ABCD method in section \ref{sec:bkgestimation:ABCD}) and post-fit MJ-VR \label{fig:results:mjcrvr}}
\end{figure}

An overview of the pre and post-fit yields in the control regions and validation regions can be seen in Table \ref{tab:analysis:PrePostYieldsCR}.
\input{lyxTables/YieldsTableCRVRonly.tex}

In the simultaneous fit described here, the multi-jet contribution in both SR-LM and SR-HM is normalised according to the normalisation factor extracted in the  Low-mass control region.  The multijet normalisation factor is compatible with unity. 
The W+jets normalisation is equaly compatible with unity within the associated uncertainties.  This is shown in Figure \ref{fig:results:wcrvr}, comparing the W-CR pre-fit with the post-fit W-VR.  A good agreement between data and \ac{SM} expectation can be seen.

\begin{figure}[htpb!]
\subfloat[WCR pre-fit]{\includegraphics[width=0.49\linewidth]{figures/PaperDraft/fig_WCR-SS.pdf}}
%\subfloat[WVR post-fit]{\includegraphics[width=0.49\linewidth]{figures/work-in-progress.pdf}}
\subfloat[WVR post-fit]{\includegraphics[width=0.49\textwidth]{figures/UpdatedThesisFigures_noATLAS/WVR_PostFit/WVR_postFit_mtmuon.pdf}}
\caption{ pre-fit WCR and post-fit WVR \label{fig:results:wcrvr}}
\end{figure}

The top normalisation is leading to a normalisation of the Monte Carlo below its nominal value,  which is motivated by the overestimation of Monte-Carlo simulation with respect to data visible in the Top control and validation regions pre-fit.  After the background-only fit,  a good agreement between \ac{SM} prediction and observed data can be seen in Figure \ref{fig:results:tcrvr}.

\begin{figure}[htpb!]
\subfloat[TCR-HM pre-fit]{\includegraphics[width=0.49\linewidth]{figures/PaperDraft/fig_TCR-ss-mvis.pdf}}
\subfloat[TVR-HM post-fit]{\includegraphics[width=0.49\linewidth]{figures/PaperDraft/fig_TVR-ss-mvis.pdf}}
\caption{pre-fit Top control region and post fit top validation region -  showing improvement in the top modelling through the fit \label{fig:results:tcrvr} \cite{AnalysisConf}}
\end{figure}

The uncertainties on the normalisation factors are largely dominated by the statistical uncertainties in the control regions.  Other contributions to the uncertainty can come from correlation effects. 
The normalisation factors and nuisance parameters describing all systematic uncertainties considered can be correlated.  A reduced correlation matrix is shown in Figure \ref{fig:analysis:corrMatrix}. 

\begin{figure}[htpb!]
 \centering
 \includegraphics[width=0.9\linewidth]{figures/C1N2SS/fitResults_250522/c_corrMatrix_RooExpandedFitResult_afterFit_Reduced_adapted.pdf}
 \caption{Reduced correlation Matrix for the C1N2SS background only fit \label{fig:analysis:corrMatrix} }
 \end{figure}

This shows a correlation between the W+jets normalisation and the Z+jets theoretical uncertainties. This correlation is due to the sizeable (yet not dominating) Z+jets contribution in the WCR and WVR due to the large available statistics in the WCR and WVR compared to other analysis regions.  A small change in the Z+jets yields due to one of the theory uncertainty related nuisance parameters leads to comparatively large changes in the Z+jets yield,  in itself leading to a smaller necessary W+jets normalisation. 

\input{lyxTables/sysTableSRs_trial_perc.tex}

With the normalisation factors extracted in the background only fit, also the systematic uncertainties included in the fit as nuisance parameters are constrained.  A post-fit breakdown of the dominating uncertainties in the signal regions is given in table \ref{tab:results:systematics}.

%Exemplary distributions in the WVR, VRF-SS-LM and TVR-C1N2SS-LM can be seen in figure \ref{fig:postFitWVR},  \ref{fig:postFitVRF} and \ref{fig:postFitTVR}.  

With this validated background estimation and systematic uncertainties in place, the signal regions were unblinded. 
Exemplary kinematic distributions of the unblinded SR-C1N2SS-LM are given in Figure \ref{fig:analysis:unblindedSRLM} as well as in Figure \ref{fig:analysis:unblindedSRHM} for SR-C1N2SS-HM.

\begin{figure}[!htpb]
\centering
\includegraphics[width=0.49\linewidth]{figures/UpdatedThesisFigures_noATLAS/LowMassunblinded/asymSRD_wDPhiOnly_TRIALsummt200_mvis_nomvisup_MT280_deltaphi_signaljets2_noLLV_nomvis_signaljetsSRLMDataTrue_SigFalse_QCDtrFalse_Mj_True_YieldsFalse_SystTrue_noRatioTrue_mc16_a_e_d_.pdf}
\includegraphics[width=0.49\linewidth]{figures/UpdatedThesisFigures_noATLAS/LowMassunblinded/asymSRD_wDPhiOnly_TRIALsummt200_mvis_nomvisup_MT280_deltaphi_signaljets2_noLLV_nomvis_SumMtTauSRLMDataTrue_SigFalse_QCDtrFalse_Mj_True_YieldsFalse_SystTrue_noRatioTrue_mc16_a_e_d_.pdf}\\
\includegraphics[width=0.49\linewidth]{figures/UpdatedThesisFigures_noATLAS/LowMassunblinded/SRL_mt2max.pdf}
\includegraphics[width=0.49\linewidth]{figures/UpdatedThesisFigures_noATLAS/LowMassunblinded/asymSRD_wDPhiOnly_TRIALsummt200_mvis_nomvisup_MT280_deltaphi_signaljets2_noLLV_nomvis_deltaPhiTauTauSRLDataTrue_SigFalse_QCDtrFalse_Mj_True_YieldsFalse_SystTrue_noRatioTrue_mc16_a_e_d_.pdf}\\
%\includegraphics[width=0.45\textwidth]{figures/C1N2SS/afterFitKinematics/SRs/310322_changedColoursAG/asymSRD_wDPhiOnly_TRIALsummt200_mvis_nomvisup_MT280_deltaphi_signaljets2_noLLV_nomvis_metSRL.pdf}
%\includegraphics[width=0.45\textwidth]{figures/C1N2SS/afterFitKinematics/SRs/310322_changedColoursAG/asymSRD_wDPhiOnly_TRIALsummt200_mvis_nomvisup_MT280_deltaphi_signaljets2_noLLV_nomvis_tau1Pt.pdf}
%\includegraphics[width=0.3\textwidth]{figures/C1N2SS/afterFitKinematics/SRs/100222_totalError_copy/asymSRD_wDPhiOnly_TRIALsummt200_mvis_nomvisup_MT280_deltaphi_signaljets2_noLLV_nomvis_tau2Pt.pdf}
\caption{ Unblinded signal region distributions for SR-C1N2SS-LM.  All normalisation factors extracted in the background-only fit are applied.  The hashed band includes statistical and sytematic errors.
\label{fig:analysis:unblindedSRLM}}
\end{figure}

\begin{figure}[!htpb]
\centering
\includegraphics[width=0.49\linewidth]{figures/UpdatedThesisFigures_noATLAS/HighMassunblinded/ditauMETSRSummthighmt2_SumMtTauSRHMDataTrue_SigFalse_QCDtrFalse_Mj_False_YieldsFalse_SystTrue_noRatioTrue_mc16_a_e_d_.pdf}
\includegraphics[width=0.49\linewidth]{figures/UpdatedThesisFigures_noATLAS/HighMassunblinded/ditauMETSRSummthighmt2_mt2maxSRHDataTrue_SigFalse_QCDtrFalse_Mj_False_YieldsFalse_SystTrue_noRatioTrue_mc16_a_e_d_.pdf}\\
%\includegraphics[width=0.45\textwidth]{figures/C1N2SS/afterFitKinematics/SRs/310322_changedColoursAG/ditauMETSRSummthighmt2_metSRH.pdf}
%\includegraphics[width=0.45\textwidth]{figures/C1N2SS/afterFitKinematics/SRs/310322_changedColoursAG/ditauMETSRSummthighmt2_tau1Pt.pdf}\\
%\includegraphics[width=0.45\textwidth]{figures/C1N2SS/afterFitKinematics/SRs/310322_changedColoursAG/ditauMETSRSummthighmt2_tau2Pt.pdf}
\caption{ Unblinded signal region kinematic distributions for SR-C1N2SS-HM.  All normalisation factors extracted in the background-only fit are applied.  The hashed band includes statistical and sytematic errors.
\label{fig:analysis:unblindedSRHM}}
\end{figure}

A breakdown of the signal region yields pre- and post-fit can be seen in table \ref{tab:results:sryields}. This shows a good agreement of the observed yields with the \ac{SM} expectation within uncertainties.  Comparing the pre- and post-fit yields, a clear reduction of the top contribution through its normalisation is visible.  Due to its negligible,  negative contribution in the SR-C1N2SS-HM,  the multi-jet contribution is set to a zero value in the fit. 

\input{lyxTables/YieldsTableSRonly.tex}

With the simultaneous likelihood fit performed in the background only fit,  a statement on the agreement of the observed events with the background-only hypothesis can be made through a simple evaluation of the model differences, taking into account the systematic uncertainties.  
An overview of this statement in terms of significances in measures of standard deviations is given in Figure \ref{fig:analysis:fitSummary}, showing a visualisation of the post-fit yields in all validation and signal regions; also including the Multiboson validation region. A good agreement between data and \ac{SM} expectation, within the uncertainties can be observed throughout all regions. 

\begin{figure}[htp!]
\centering
\includegraphics[width=0.9\linewidth]{figures/C1N2SS/fitResults_250522_noATLAS/histpull_atlasbkgonlyunblinded_250522VR.pdf}
\caption{Pull Plot: Summary of all validation and signal regions included in the background only fit and the significance of the difference of background estimation compared to observed data  \label{fig:analysis:fitSummary}}
\end{figure}

\FloatBarrier
\section{Interpretation}
\label{sec:analysis:interpretation}
With no significant excesses observed in the signal region, the observed results were interpreted in a model-dependent as well as model-independent approach described in the following sections. 

\subsection{Model dependent interpretation}
As described in previous sections (sec.  \ref{sec:analysis:intro} and \ref{sec:analysis:sroptimisation}), the signal regions were designed to have sensitivity to the simplified model describing the production of a \Cone and \Ntwo .  Using the CLs approach described in section \ref{sec:analysis:stats},  a hypothesis test is performed for mass parameter value highlighted in section \ref{sec:analysis:samples}.  Mass points with a CLs value below 0.05 are excluded.  This can be seen through the exclusion contours in figure \ref{fig:analysis:SRLMcontour}, \ref{fig:analysis:SRHMcontour} and \ref{fig:analysis:combinedcontour}.  The CLs value can be calculated for each signal region independently, as well as the statistical combination of both regions. 
The contours are shown in the \None versus \Cone mass plane,  with the stau mass parameter $x=0.5$. The expected limit as dashed line is describing the limit up to which all enclosed mass points are excluded at a 95\% confidence level,  assuming that the observed data is equivalent to the \ac{SM} expectation.  This is a measure of the designed analysis sensitivity and is accompanied by the 'yellow band', visualising how the expected limit shifts when varying the associated experimental uncertainties by $\pm 1 \sigma_{\text{exp}}$.  This experimental uncertainty band includes all experimental and theoretical systematic uncertainties on the background estimations as well as on the signal prediction,  apart from an inclusive cross section uncertainty on the signal prediction. 
This uncertainty purely due to the inclusive cross section uncertainty on the signal expectation is visualised in the dashed red band around the observed limit in Figure \ref{fig:analysis:combinedcontour}. 
 
\begin{figure}[htpb!]
\centering
\subfloat[SR-C1N2SS-LM \label{fig:analysis:SRLMcontour}]{\includegraphics[width=0.5\linewidth]{figures/UpdatedContours_130622_noATLAS/contourLow_testing_160622_NOM_onefile.pdf}}
\subfloat[SR-C1N2SS-HM \label{fig:analysis:SRHMcontour}]{\includegraphics[width=0.5\linewidth]{figures/UpdatedContours_130622_noATLAS/contourHigh_testing_160622_NOM_onefile.pdf}}\\
\subfloat[Combined SR \label{fig:analysis:combinedcontour}]{\includegraphics[width=0.7\linewidth]{figures/UpdatedContours_130622_noATLAS/contourSS_pyHF_160622.pdf}
}
\caption{Exclusion limits in the SR-C1N2SS-LM  (a),  SR-C1N2SS-HM (b) as well as both signal regions statistically combined (c)} 
\end{figure}
The region SR-C1N2SS-LM (Fig. \ref{fig:analysis:SRLMcontour}) is able to exclude mass points with low masses and towards the kinematic diagonal.  The region SR-C1N2SS-HM is pushing the analysis sensitivity towards higher chargino masses (Fig.  \ref{fig:analysis:SRHMcontour}). 
The statistical combination of SR-C1N2SS-LM and SR-C1N2SS-HM is decreasing the associated statistical uncertainty and therefore further increasing the analysis sensitivity.  
As can be seen both in the SR-C1N2SS-HM as well as in the combined exclusion contour, for high \Cone masses the observed exclusion is weaker than the expected exclusion reach.  This is due to a slight fluctuation in observed data in the SR-C1N2SS-HM compared to its nominal \ac{SM} expectation.  This fluctuation is well within the systematic uncertainties (within the yellow band). 
Masses of the lightest chargino up to 960 GeV for massless lightest neutralinos were excluded.  Mass differences between the \Cone and \Ntwo as little as 30 GeV have been excluded for a \Cone mass of 80 GeV.  

\FloatBarrier
\subsection{Model independent interpretation}

The most general result of any \ac{BSM} search is offered through model-independent interpretations.  This neglects any asssumption of the shape or yield of a signal model and offers an upper limit on the visible cross section. 
In both signal regions,  an upper-limit,  model-independent,  fit is performed to extract the highest possible cross section that a \ac{BSM} signal could have in order for the analysis to have noticed the contribution within 95\% confidence level. 
This is measured through the maximum number of signal events that can populate the signal regions in order to still be in agreement with the background-only hypothesis. 

\begin{table}[htpb!]
\centering
\setlength{\tabcolsep}{0.0pc}
\begin{tabular*}{\textwidth}{@{\extracolsep{\fill}}lcccccc}
\noalign{\smallskip}\hline\noalign{\smallskip}
{\textbf{Signal channel}}                        & $\langle\epsilon{ \sigma}\rangle_{ obs}^{95}$[fb]  & $\langle\epsilon{ \sigma}\rangle_{ exp}^{95}$[fb] &  $S_{ obs}^{95}$  & $S_{ exp}^{95}$ & $CL_{B}$ & $p(s=0)$ ($Z$)  \\
\noalign{\smallskip}\hline\noalign{\smallskip}
%%
 SR-C1N2SS-LM    & $0.03$ &  $0.03$ & $4.7$ & $ { 4.6 }^{ +1.8 }_{ -0.5 }$ & $0.56$ & $ 0.41$~$(0.23)$ \\%
 SR-C1N2SS-HM  & $0.04$ &  $0.04$ & $6.2$ & $ { 5.0 }^{ +2.1 }_{ -0.9 }$ & $0.73$ & $ 0.39$~$(0.29)$ \\%
\noalign{\smallskip}\hline\noalign{\smallskip}
\end{tabular*}
\caption[Breakdown of upper limits.]{
Listed are the observed and expected 95\% CL upper limit on the visible cross section ($\langle\epsilon\sigma\rangle_{ obs}^{95}$, $\langle\epsilon\sigma\rangle_{ exp}^{95}$ respectively),  the upper limit on the number of signal events
($S_{ obs/exp}^{95}$) as well as the confidence level for the background-only hypothesis $CL_B$.  The last column describes the discovery $p$-value. 
\label{tab:analysis:upperlimit}}
\end{table}

The results of the model-independent upper limits are shown in table \ref{tab:analysis:upperlimit}.  Listed are the observed and expected 95\% CL upper limit on the visible cross section ($\langle\epsilon\sigma\rangle_{ obs}^{95}$, $\langle\epsilon\sigma\rangle_{ exp}^{95}$ respectively),  the upper limit on the number of signal events
($S_{ obs/exp}^{95}$) as well as the confidence level for the background-only hypothesis $CL_B$.  The last column describes the discovery $p$-value. 
The largest difference between observed and expected upper limit on the signal events can be seen in SR-C1N2SS-HM, with an expected $S_{ exp}^{95}$ of $5.0^{+ 2.1}_ {-0.9}$,  whereas the observed limit reaches $6.2$.  This is due to the slight upward fluctuation in data in the SR-C1N2SS-HM,  allowing for a larger potential signal contribution. 

%\begin{table}
%\centering
%\setlength{\tabcolsep}{0.0pc}
%\begin{tabular*}{\textwidth}{@{\extracolsep{\fill}}lcccccc}
%\noalign{\smallskip}\hline\noalign{\smallskip}
%{\textbf{Signal channel}}                        & $\langle\epsilon{ \sigma}\rangle_{ obs}^{95}$[fb]  & $\langle\epsilon{ \sigma}\rangle_{ exp}^{95}$[fb] &  $S_{ obs}^{95}$  & $S_{ exp}^{95}$ & $CL_{B}$ & $p(s=0)$ ($Z$)  \\
%\noalign{\smallskip}\hline\noalign{\smallskip}
%%%
% SR-C1N2SS-LM    & $0.03$ &  $0.03$ & $4.7$ & $ { 4.6 }^{ +1.8 }_{ -0.5 }$ & $0.56$ & $ 0.41$~$(0.23)$ \\%
% SR-C1N2SS-HM  & $0.04$ &  $0.04$ & $6.2$ & $ { 5.0 }^{ +2.1 }_{ -0.9 }$ & $0.73$ & $ 0.39$~$(0.29)$ \\%
%\noalign{\smallskip}\hline\noalign{\smallskip}
%\end{tabular*}
%\caption[Breakdown of upper limits.]{
%Listed are the observed and expected 95\% CL upper limit on the visible cross section ($\langle\epsilon\sigma\rangle_{ obs}^{95}$, $\langle\epsilon\sigma\rangle_{ exp}^{95}$ respectively),  the upper limit on the number of signal events
%($\langle\epsilon\sigma\rangle_{ obs/exp}^{95}$) as well as the confidence level for the background-only hypothesis $CL_B$.  The last column describes the discovery $p$-value. 
%\label{table.results.exclxsec.pval.upperlimit.SRLowMass}}
%\end{table}


