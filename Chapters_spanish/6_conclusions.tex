\chapter*{Conclusiones}
\addcontentsline{toc}{part}{Conclusiones}
\markboth{}{Conclusiones}
% \epigraph{\emph{Don't let anyone rob you of your imagination, your creativity, or your curiosity.}} {Mae Jemison}




La enorme cantidad de datos producidos por el \ac{LHC} a lo largo de su operación ha sido crucial para la realización de numerosas investigaciones en física de altas energías, cuyos resultados se encuentran en la frontera del conocimiento actual en el campo. Gracias a las altas energías de colisión y la capacidad para generar un gran volumen de eventos, el \ac{LHC} ofrece condiciones únicas para llevar a cabo experimentos que investigan regiones en la frontera energética hasta ahora inexploradas. Entre sus principales experimentos se encuentra el detector \ac{ATLAS}, en el que se enmarca la presente tesis. Este detector ha sido clave en búsquedas de nuevas partículas, destacándose el histórico descubrimiento del bosón de Higgs en 2012 que permitió completar el \ac{SM} de las partículas elementales y sus interacciones. Las investigaciones realizadas en esta tesis están motivadas por algunas de las cuestiones que el \ac{SM} no ha podido responder, entre las que se incluyen el problema de jerarquías y la existencia de tres familias de fermiones.

La producción de fotones prompt en colisiones \pp del \ac{LHC} es clave en el programa de física del experimento \ac{ATLAS}, tanto para estudios de \ac{pQCD} como para búsquedas \ac{BSM} que incluyen fotones aislados en el estado final, como es el caso de esta tesis. Uno de los grandes desafíos para estas investigaciones consiste en lograr una identificación de fotones con alta eficiencia y pureza, debido a que algunos jets, cuya producción es dominante en las colisiones del \ac{LHC}, pueden ser identificados erróneamente como fotones por sus características similares. En \ac{ATLAS}, el proceso de identificación de fotones se basa en estudiar la lluvia electromagnética de partículas en el calorímetro mediante variables derivadas de datos y simulaciones. Dado que las simulaciones no coinciden exactamente con los datos reales, una parte de esta tesis se centró en corregir esas discrepancias mejorando el método actual de corrección, denominado \acf{FF}~\cite{ATLAS-QT-Sili}, e investigando un enfoque alternativo mediante el uso de variables de nivel inferior en la reconstrucción de señales en el detector, con la ventaja que permite optimizar simultáneamente todas las variables de identificación de fotones. Este método es de gran interés para la colaboración \ac{ATLAS} permitiendo la utilización de algoritmos de Machine Learning para la identificación de fotones, en particular en el contexto del futuro High-Luminosity \ac{LHC}. El método mejorado de los \acp{FF} brinda una performance superior de la identificación de fotones y es actualmente utilizado por todos los análisis de la colaboración.

El trabajo central de esta tesis es la búsqueda de nueva física~\cite{ATLAS-PhotonJetResonances-Run2-NOTE} utilizando el conjunto completo de datos del Run-2 del \ac{LHC} a una energía de centro de masa de \(\sqs = 13~\TeV\), recolectados por el detector \ac{ATLAS} durante los años 2015 y 2018, correspondiendo a una luminosidad integrada de \(140.01~\ifb\). La misma está motivada por modelos que apuntan a resolver dos grandes interrogantes del \ac{SM}: el por qué de las tres familias de fermiones y el problema de las jerarquías. El modelo de \acf{EQ}, el cual provee una solución al primero de ellos, predice que los fermiones no son partículas fundamentales sino que son estados ligados de partículas más fundamentales que decaen rápidamente en un bosón y un fermión. Por otro lado, modelos con dimensiones extra proponen soluciones al problema de jerarquías que resultan en la existencia de \acf{QBH} que decaen en un número bajo de partículas. En esta tesis, se consideraron aquellos modelos en los cuales los \ac{EQ} y los \ac{QBH} decaen en un fotón y un quark. En el caso de los modelos de \ac{EQ}, se consideraron diferentes acoplamientos (\(f\)) de los mismos con los bosones del \ac{SM} así como también diferentes sabores, es decir, se consideran por separado los procesos de producción de \qstar (\(u^*/d^*\)), \cstar y \bstar, siendo para el caso de \cstar el primer estudio que se realiza en experimentos del \ac{LHC}. Los modelos de Randall-Sundrum y ArkaniHamed-Dimopoulos-Dvali, que proponen una y seis dimensiones extra, respectivamente, son estudiados en el marco de la producción de \ac{QBH}. Dado que la forma de la masa invariante del par fotón+jet (\myj) de procesos del \ac{SM} decae suavemente, la búsqueda se basa en la identificación de excesos locales, o resonancias, en el espectro de masa. Para ello, se definió una región de señal inclusiva, la que a su vez se separa en tres regiones ortogonales basadas en el sabor del jet. El modelado del fondo consiste en ajustes funcionales a los propios datos experimentales. La posible diferencia entre el fondo y los datos, que llevaría a un descubrimiento, es cuantificada en primer lugar mediante un valor-p y posteriormente por el algoritmo BumpHunter, que realiza una búsqueda del exceso más significativo. En todas las regiones de señal consideradas se encontró que el fondo reproduce correctamente los datos sin evidencia estadística de un exceso sobre el mismo. Se establecieron entonces límites superiores en la sección eficaz visible a un \(95\%\) \ac{CL} considerando resonancias genéricas dadas por formas Gaussianas, así como límites en los modelos de \ac{EQ} y \ac{QBH}, excluyendo posibles parámetros de las teorías a 95\% \ac{CL}. Para el caso de \ac{EQ} con \(f = 1\), se excluyen modelos con \qstar, \cstar y \bstar con masas de hasta \(6160\), \(3392\) y \(2469~\GeV\), respectivamente. Finalmente, los modelos de \ac{QBH} con una (seis) dimension(es) extra(s) son excluidos hasta masas de \(5349~\GeV\) (\(7581~\GeV\)). Los resultados obtenidos en esta tesis constituyen los límites más rigurosos en todos los modelos considerados hasta la fecha. Asimismo, se destaca que en el caso de los \cstar, se trata de los primeros resultados obtenidos en experimentos del \ac{LHC}, lo que implica un marcado avance tanto del punto de vista teórico como experimental.