\chapter*{Introducci\'on}
\addcontentsline{toc}{chapter}{Introducci\'on}
\markboth{}{Introducci\'on}

This thesis presents a search for new phenomena in high-mass final states with a photon and a jet in proton-proton (\pp) collisions at a centre-of-mass energy of 13 TeV using data collected by the \acs{ATLAS} detector. \acs{ATLAS} (“A Toroidal LHC ApparatuS”) is one of the two general-purpose detectors at the Large Hadron Collider (\acs{LHC}) and the biggest multi-purpose particle detector ever built. It is used to investigate a wide range of physics, from Standard Model (\acs{SM}) measurements, such as precision tests of quantum chromodynamics or study of the properties of the Higgs boson, to the search of new phenomena like extra dimensions and dark matter candidates. The LHC, built by the European Organization for Nuclear Research (\acs{CERN}) and installed in a 27-kilometre circular tunnel, is the world's largest and most powerful particle collider. This machine is capable of colliding energetic beams of protons (or heavier nuclei) at rates upward of millions per second. The precision and high beam energy of the LHC allow to explore the tera-electronvolt scale, an energy range never before achieved in a particle collider.



The thesis is divided into three parts. The first part describes the theory background and motivations of the work, in which the first chapter describes the Standard Model (SM) showing the excellent agreement there is between the theory and the experimental data. It also shows the actual problems of the SM, which motivates the search for New Physics.

The second part describes the experiment. CHAPTER describes the ATLAS experiment, with each sub-detector in detail, to finally describes how the samples for data analysis are produced. The reconstruction of the different physics objects is explained in CHAPTER. At the end of the second part, a special chapter is dedicated to the photon identification, where a new method for correcting the shower shapes is explained and detailed.

The third part of the thesis is dedicated to the photon+jet resoanance search.
In CHAPTER, the general strategy and motivation of the search is discussed. The samples generation is discussed in CHAPTER, where the final event selection is given in CHAPTER.
The methods for the background estimation as well as background modeling is given in CHAPTER AND CCHAPETER.
The systematic uncertainties are discussed in CHAPTER, and finally in CHAPTER and CHAPTER, the statistical analysis with the results are shown.

Finally, the conclusions of the work are present in CHAPTER.


%\epigraph{\emph{The journey, not the destination matters.}}{Thomas S. Eliot}


%Provide preliminary background information that puts your research in context (Why?)
%
%Clarify the focus of your study (What?)
%
%Point out the value of your research(including secondary research)!! (What gain?)
%
%Specify your specific research aims and objectives 

%The reader needs to know why your research is worth doing


%maybe want to motivate why looking at run-2 data? 

%At the time of writing
%between run 2 and run 3
%at time of writing,  run-3 data-taking is just about to start.. 
%important to use previous dataset to gauge limitations of searches
%--> crucial to contribute to the overall runnning of the experiment
%--> study and understand trigger 
%The result is striking. Two --> WIMP miracle!
%seemingly unrelated problems – the Higgs unnaturalness (rooted in the quantum structure
%of the particle world at distances of 10−20 meters and below) and the nature of dark matter
%(observed from galactic distances of 1020 meters to the largest scales in the universe)
%the science historian Thomas Kuhn
%We are confronted with the need to reconsider the guiding principles that have been used for decades to address the most fundamental questions about the physical world. These are symptoms of a phase of crisis.
%Greek krisis, which means “decisive moment”, “turning point”,
%privilege of the opportunity for an upcoming paradigm change.

% Within the last decades of particle physics research there have been many milestones and discoveries, narrowing in on the building blocks of matter and leading to the development of the Standard Model (\acs{SM}) of particle physics as we know it today. 
% This has been achieved through extraordinary efforts and interplay between experimental measurements unravelling hints and evidence of new particles,  and theoretical effort of tying this to an overarching theory and achieving more precise predictions:
% from the discovery of the electron in 1897 \cite{ElectronThomson},  to the discovery of the tau lepton in 1975 \cite{TauDiscovery},  to the most recent success of the theoretical predictions lying in the experimental discovery of the Higgs boson \cite{ATLASHiggsDiscovery,CMSHiggsDiscovery}.  In the last ten years since the Higgs boson's discovery,  its properties have been measured to high precision \cite{HiggsReviewATLAS10years} and the Standard Model’s predictions have been put to stringent tests. 
% In various measurements, tensions with predictions of the Standard Model have been appearing, hinting towards a larger underlying theory. The limitations of the Standard Model have become more and more evident. The Hierarchy problem,\cite{SUSYPrimer}, questioning the difference in scales between the Higgs mass and the Planck scale, raising the need for unnatural fine-tuning, is only one of the hints for the need for a larger principle. The evolution of our universe into a matter-dominated universe is so far unexplained, with no mechanisms within the Standard Model to sufficiently generate this asymmetry with anti-matter \cite{WMAP, Thomson,Sakharov}. Lastly, one of the most striking hints connects the limitations of the theoretical model describing the smallest building blocks in our universe to the largest structures known to humankind. The presence of Dark Matter \cite{Zwicky,RubinKent,RotationCurves,Planck,BulletCluster, OtherMergingClusters} in galaxies and galaxy clusters can not be explained through a composition of Standard Model particles. 

% An additional symmetry between bosons and fermions called \ac{SUSY} could offer solutions to many of these limitations. Not only could \ac{SUSY}  avoid the need for fine-tuning of the Higgs mass, but also \acp{WIMP} predicted by SUSY could make up a component of Dark Matter. This striking connection of two problems at length scales varying from the size of a fundamental particle to galaxies and galaxy clusters is known as the WIMP miracle. \\
% Supersymmetric particles have been searched for in many ways at the Large-Electron-Positron collider, the Large-Hadron-Collider as well as through a variety of non-collider experiments. Up to the moment of writing this thesis, there has been no direct detection of Dark Matter particles and no evidence of supersymmetric particles at colliders. 

% According to definitions of science historian Thomas Kuhn and discussion thereof of Gian Francesco Giudice (\cite{Kuhn,Giudice}) particle physics can currently be described to be in a period of "krisis".  Krisis has to be understood in its original meaning - a period of change and anticipation. This period can be frustrating and confusing, with a lack of direction to a new underlying principle,  but should be seen as a privilege.  A period preceding a paradigm change,  with room for creativity for new ideas but also the need for diligent exploration of limitations in current experiments.

% At the time of writing, a new data-taking period at the LHC has just begun. In preparation for this new data taking, it is crucial to thoroughly analyse the Run-2 data set, find its limitations and uncovered areas of new physics searches in order to prepare for the new challenges to come. 

% In this thesis, a search for \ac{SUSY} has been performed, looking for a production of the lightest chargino and next to lightest neutralino (supersymmetric partners of SM gauge bosons), decaying via a scalar tau lepton into a final state with hadronically decaying tau leptons. This search has been performed with data collected by the ATLAS detector at the LHC, as part of the ATLAS Collaboration. This final state with hadronically decaying tau leptons belongs to the ‘paths less walked’ within the ATLAS Collaboration, due to its challenging reconstruction. This offers an interesting window to determine and overcome the limitations of the ATLAS Collaboration’s search program for \ac{SUSY}. This analysis has been the author’s full responsibility. 
% Next to this main effort within this thesis, the performance of electron triggers within ATLAS have been studied as part of the author's qualification task as well as continued commitment to ensure the successful operation and good performance of ongoing ATLAS data-taking.

% The structure of this thesis is as follows: A brief overview of the theoretical concepts of the SM as well as its limitations,  motivating \ac{SUSY} is given in Chapter \ref{ch:theory}. This is followed by a conceptual description of LHC proton-proton collisions and the ATLAS detector in chapter \ref{ch:expsetup}. Further details on the data collection and reconstruction of collision events as well as the simulations used to study the events is given in Chapter \ref{ch:DAQ}. A detailed view on the electron trigger and its performance is given in Chapter \ref{ch:trigger}. The search for supersymmetric gauge bosons in all its necessary details is given in Chapter \ref{ch:analysis}. 