\chapter*{Introducción}
\addcontentsline{toc}{chapter}{Introducción}
\markboth{}{Introducción}

Nuestra mejor comprensión actual de la física de partículas viene dada por la \ac{SM}, una teoría que explica con éxito una amplia gama de resultados experimentales y predijo con precisión muchos fenómenos físicos diferentes. También, predijo la existencia de nuevas partículas, como el bosón de Higgs descubierto en 2012 por las colaboraciones \acs{ATLAS} y \acs{CMS}, que llevó a la concesión del Premio Nobel a François Englert y Peter Higgs al año siguiente. A pesar de su notable éxito, se sabe que el \ac{SM} es incompleto ya que no puede explicar una serie de observaciones experimentales, como la abrumadora evidencia astrofísica y cosmológica de la materia oscura, el problema de jerarquías, por qué solo hay tres familias de fermiones, entre otros. En las últimas décadas, surgieron muchas teorías para la nueva física \ac{BSM}, como \ac{SUSY}, que proporcionaron marcos teóricos bien motivados y prometedores para ampliar nuestra comprensión fundamental de la física de partículas y mejorar las deficiencias del \ac{SM}. Sin embargo, ninguna de las numerosas búsquedas de nueva física que se han llevado a cabo en los colisionadores de partículas en los últimos años ha podido aportar pruebas directas de la existencia de nuevas partículas o fuerzas, tal y como predecían estas teorías.

Una gran variedad de estos nuevos modelos teóricos predicen la existencia de partículas a altas energías. Para explorar estas regiones, especialmente la escala \tev, en el laboratorio \acused{CERN}\ac{CERN} se construyó el \ac{LHC}~\cite{LHC-Machine}. Instalado en un túnel circular de 27 kilómetros, es el mayor y más potente colisionador de partículas del mundo. Esta máquina es capaz de colisionar haces energéticos de protones a velocidades superiores a millones por segundo. La precisión y la elevada energía de los haces del \ac{LHC} permiten explorar energías en la escala del \tev, un rango energético nunca antes alcanzado en un colisionador de partículas. El \ac{LHC} colisiona protones en cuatro puntos de interacción, donde se encuentran los 4 experimentos principales del \ac{LHC}: \acused{ATLAS}\ac{ATLAS}, \acused{CMS}\ac{CMS}, \acused{LHCb}\ac{LHCb} y \acused{ALICE}\ac{ALICE}.
Entre los años 2015 y 2018 tuvo lugar un periodo de toma de datos denominado Run-2, en el que se colisionaron protones a \(\sqs=13~\tev\), y recogiendo un total de luminosidad integrada de \(140.01~\ifb\). En 2022 comenzó el \ac{LHC} Run-3, en el que se aumentó la energía del centro de la masa a \(\sqs=13.6~\tev\), y a finales de 2024 \ac{ATLAS} llego a recolectar \(183~\ifb\) de datos.

Uno de los experimentos más importantes del \ac{LHC} es el \ac{ATLAS} (\acl{ATLAS}), un detector multipropósito diseñado para realizar tanto medidas de precisión dentro del \ac{SM} como búsquedas de nuevos fenómenos asociados a la física \ac{BSM}. El detector \ac{ATLAS} está compuesto por distintos subdetectores que desempeñan diferentes papeles en la reconstrucción de las partículas en colisión. El \acl{ID} se encarga de medir las trazas de partículas cargadas, los calorímetros miden las deposiciones energéticas de fotones, electrones y diferentes hadrones, y finalmente el \acl{MS} permite medir las trayectorias de los muones. Entre ellos se encuentra un potente sistema magnético, que curva la trayectoria de las partículas cargadas. Por último, el detector \ac{ATLAS} dispone de un preciso sistema trigger que filtra los eventos de poco interés, reduciendo así la frecuencia del flujo de datos.
En cualquier experimento de física de altas energías, es habitual trabajar con simulaciones, tanto para los procesos conocidos del \ac{SM}, como para comprender las formas de las nuevas señales predichas por los escenarios \ac{BSM}. Esto añade otro grado de complejidad al experimento, ya que se requiere que las simulaciones describan de forma excelente los procesos físicos reales que se obtendrían de los datos reales.

La producción de fotones prompt a partir de colisiones \pp en el \ac{LHC} constituye una parte clave del programa de física \ac{ATLAS}, ya sea para medidas precisas de observables \ac{QCD}, o porque varios escenarios \ac{BSM} implican tener fotones prompt aislados en el estado final. Sin embargo, el principal proceso que tiene lugar en las colisiones \pp es la producción de jets, y a veces uno de estos jets tiene caracter'isticas muy similares a las que tendría un fotón, por lo que este jet se identifica erróneamente como un fotón. El proceso de identificación en \ac{ATLAS} constituye uno de los ingredientes principales en cualquier análisis de física. En el caso de los fotones, esta identificación se lleva a cabo estudiando la lluvia \ac{EM} iniciada por las partículas en el calorímetro utilizando diversas variables que describen la forma de éstas, y como se anticipó, se lleva a cabo utilizando los datos y las simulaciones. Sin embargo, se observó que la simulación no predecía correctamente los datos, lo que conducía a resultados incongruentes.
Una de las principales tareas de esta tesis es la corrección de las variables utilizadas para la identificación de fotones. El método actual para corregir las variables se denomina \ac{FF}, que se mejoró drásticamente en este trabajo. Además, se estudió otro enfoque, en el que se pueden realizar modificaciones en variables de nivel inferior para as'i corregir simultáneamente todas las variables utilizadas para identificar fotones.

En el párrafo anterior se mencionó que los fotones prompt son de gran importancia para las búsquedas \ac{BSM}. En particular, en el estado final fotón+jet, la masa invariante sigue una forma descendente muy suave, proporcionando un escenario excelente para las búsquedas de bump, donde se pueden buscar diferentes partículas que decaen a un par fotón+jet. Dos de los modelos teóricos que pretenden dar respuesta a las diferentes carencias del \ac{SM} predicen la existencia de este tipo de partículas. El primero da una explicación de por qué existen tres familias de fermiones, y propone que los quarks no son partículas fundamentales sino estados ligados de otras más fundamentales que experimentan una fuerza desconocida. Entonces, deberían observarse estados de \ac{EQ} (\qstar) en colisiones \pp en el \ac{LHC} dependiendo de si el valor de la escala de composición \(\Lambda\) es menor que la energía del centro de masa. Estos \ac{EQ} decaerían en un par de fotón y jet, dejando una \textit{bump} en la distribución de \myj alrededor de la masa del \ac{EQ}. El segundo modelo, con la introducción de dimensiones extra, intenta proponer una solución para el problema de jerarquías. Ciertos tipos de modelos de dimensiones extra predicen que la escala fundamental de Planck \(m_P\) en las \(4 + n\) dimensiones (siendo \(n\) el número de dimensiones espaciales extra) está en la escala \tev, y por tanto accesible en colisiones \pp en la \(\sqs=13~\tev\).
En tal escala \TeV, a partir de una masa umbral \(m_{\text{th}}\), se podr'ian producir \ac{QBH} en colisiones \pp del \ac{LHC} y luego decaer en un pequeño número de partículas de estado final incluyendo pares fotón-quark/gluón antes de que sean capaces de termalizarse. En este caso podría observarse una amplia estructura resonante justo por encima de \(m_{\text{th}}\) sobre la distribución de \myj del \ac{SM}. Bajo esta teoría se estudian dos modelos particulares, que proponen diferentes números de dimensiones extra: el modelo RS1 de Randall-Sundrum propone un total de 5 dimensiones espacio-temporales, y el modelo ADD de Arkani-Hamed, Dimopoulos y Dvali, que cuenta con un total de 10 dimensiones espacio-temporales.
Por último, dada la suavidad de la distribución \myj, es posible realizar una búsqueda agnóstica del modelo sobre este fondo, en la que se considera que la señal sigue una resonancia de forma gaussiana. Este tipo de búsqueda proporciona una interpretación más general del estudio, ya que permite comparar con cualquier modelo de teoría que proponga una resonancia de forma Gaussiana.


Estudios previos similares de b\'usquedas de resonancias en el estado final de \gammajet considerando los mismos modelos teóricos han sido llevados a cabo. estableciendo as'i límites superiores a las teorías. El último resultado de \ac{ATLAS} utilizó \(36.7~\ifb\) y excluyó los modelos de \qstar con masas de hasta \(5.3~\tev\), los \ac{QBH} del tipo RS1 hasta \(4.4~\tev\) y los \ac{QBH} del tipo ADD hasta \(7.1~\tev\). Por otro lado, \ac{CMS} también realizó estudios similares utilizando \(138~\ifb\), donde estudiaron modelos de \ac{EQ} que separaron en resonancias livianas (\qstar) y pesadas (\bstar), estableciendo límites m\'as estrictos de \(6.0~\tev\) y \(2.2~\tev\), respectivamente. \ac{CMS} también estudió los modelos de \ac{QBH} ADD y RS1, en los que los límites superiores de las masas se extienden hasta \(7.5~\tev\) y \(5.2~\tev\), respectivamente.

El trabajo principal de esta tesis es la búsqueda de resonancias de alta masa en el estado final fotón+jet. La búsqueda se realiza utilizando el conjunto completo de datos Run-2 recolectados a una energía de centro de masa de \(\sqs=13~\tev\), utilizando un total de \(140.01~\ifb\). Se estudian los modelos \ac{EQ} y \ac{QBH}, así como la búsqueda de señales genéricas de forma gaussiana.
En particular, para los modelos de \acp{EQ}, se separan las señales en \qstar (\(u^*/d^*\)), \cstar y \bstar, siendo este trabajo el primero en el \ac{LHC} considerando el sabor charm, gracias a un novedoso algoritmo de tagging de sabores, que proporciona un excelente rendimiento.




La tesis se divide en cuatro partes.
La \Part{\ref{part:theory}} describe el marco teórico y las motivaciones del trabajo, donde se describe brevemente el \ac{SM} y se discuten los dos modelos teóricos que motivan la b\'usqueda de f'isica \ac{BSM} de este trabajo.

La \Part{\ref{part:exp_setup}} contiene dos capítulos en los que el primero (\Ch{\ref{ch:atlas}}) describe detalladamente el \ac{LHC} y el detector \ac{ATLAS}, haciendo hincapié en las distintas partes del detector. En el \Ch{\ref{ch:objects}}, se analizan los métodos utilizados para la reconstrucción e identificación de los distintos objetos f'isicos.

En la \Part{\ref{part:pid}} se explican los diferentes algoritmos utilizados para la identificación de fotones. Primero, en el \Ch{\ref{ch:pid_ss}}, se discuten las variables utilizadas para la identificación de fotones, cómo se optimiza y cómo se realizan las medidas de eficiencia de la identificación. Luego, en el \Ch{\ref{ch:ss_corrections}}, se presentan los métodos diferentes, obtenidos en esta tesis, para corregor las variables simuladas que se usan para realizar la identificación de fotones.

Finalmente, la \Part{\ref{part:search}} presenta la búsqueda de resonancias de \gammajet en el rango de altas masas. El \Ch{\ref{ch:strategy}} comienza discutiendo la estrategia general de análisis y establece los métodos estadísticos que se utilizarán. A continuación, en el \Ch{\ref{ch:samples}}, se describen las muestras que se utilizan tanto para los modelos teóricos como para los fondos del \ac{SM}. La selección de eventos y las definiciones de las regiones de señal que se utilizan en la búsqueda se presentan en el \Ch{\ref{ch:evt_selection}}. El modelado de la señal y las incertezas sistemáticas teóricas y experimentales se describen en el \Ch{\ref{ch:signals}}, que constituyen uno de los puntos más importantes y desafiantes de cualquier búsqueda. Adem\'as, es crucial tener un excelente conocimiento de los fondos del \ac{SM}. Para ello, en el \Ch{\ref{ch:bkg}}, se estudian los dos fondos principales, donde se utiliza un modelo funcional para modelar el fondo total del \ac{SM}. Por último, en el \Ch{\ref{ch:results}} se presentan y discuten los resultados obtenidos en esta búsqueda.