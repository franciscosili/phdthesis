\chapter*{Introducción}
\addcontentsline{toc}{chapter}{Introducción}
\markboth{}{Introducción}


Nuestra comprensión actual de la física de partículas está sustentada en el \ac{SM} de las partículas elementales y sus interacciones fundamentales, una teoría que explica con éxito una amplia gama de resultados experimentales y que predice con alta precisión diversos fenómenos físicos. El contenido de partículas del \ac{SM} incluye al bosón de Higgs descubierto en 2012 por las colaboraciones \acs{ATLAS} y \acs{CMS}, que llevó al otorgamiento del Premio Nobel a François Englert y Peter Higgs al año siguiente. A pesar de su éxito remarcable, el \ac{SM} es incompleto ya que no puede explicar una serie de observaciones experimentales, como por ejemplo  la evidencia astrofísica y cosmológica de la materia oscura, el problema de jerarquías relacionado a la gran discrepancia entre las escalas de las interacciones débil y la gravedad, por qué solo hay tres familias de fermiones, entre otras. En las últimas décadas surgieron una gran variedad de teorías de física \ac{SM} que proporcionan marcos teóricos bien motivados y prometedores y, en paralelo, se realizaron numerosas investigaciones experimentales en la frontera de energía en los aceleradores más potentes del mundo, con la finalidad de ampliar nuestra comprensión fundamental de la física y mejorar las deficiencias del \ac{SM}. Sin embargo, ninguna de las búsquedas de nueva física que se han llevado a cabo en los colisionadores de partículas en los últimos años ha podido aún aportar pruebas directas de la existencia de nuevas partículas o fuerzas, resultando en el establecimiento de límites muy estrictos en las diversas teorías con extensiones al \ac{SM}.

Para explorar las regiones de altas energías, especialmente la escala \TeV, en el laboratorio \acused{CERN}\ac{CERN} se construyó el \ac{LHC}~\cite{LHC-Machine}. Instalado en un túnel circular de 27 kilómetros, el \ac{LHC} es el mayor y más potente colisionador de partículas del mundo. Esta máquina es capaz de colisionar haces energéticos de protones acelerados a velocidades próximas a la velocidad de la luz. La precisión y la elevada energía de los haces del \ac{LHC} permiten explorar energías en la escala del \TeV, un rango energético nunca antes alcanzado en un colisionador de partículas. El \ac{LHC} colisiona protones en cuatro puntos de interacción donde se encuentran los 4 experimentos principales del \ac{LHC}: \acused{ATLAS}\ac{ATLAS}, \acused{CMS}\ac{CMS}, \acused{LHCb}\ac{LHCb} y \acused{ALICE}\ac{ALICE}. Entre los años 2015 y 2018 tuvo lugar un periodo de toma de datos denominado Run-2, en el que las colisiones entre protones alcanzaron energía de centro de masa de \(\sqs = 13~\TeV\), recogiendo un total de luminosidad integrada de \(140.01~\ifb\) de datos. En 2022 comenzó el \ac{LHC} Run-3, en el que se aumentó la energía del centro de la masa a \(\sqs = 13.6~\TeV\) y la luminosidad instantánea lo que permitió que, ya a finales de 2024, el experimento \ac{ATLAS} recolectara \(183~\ifb\) de datos.

Uno de los experimentos más importantes del \ac{LHC}, en cuyo marco se llevó a cabo esta tesis, es el \ac{ATLAS}, un detector multipropósito diseñado para realizar tanto medidas de precisión de procesos del \ac{SM} como búsquedas de nuevos fenómenos asociados a la física \ac{BSM}. El detector \ac{ATLAS} está compuesto por distintos subdetectores, diseñados para poder determinar las propiedades de las partículas que se producen en cada colisión. El \acl{ID} se encarga de medir las trazas de partículas cargadas, los Calorímetros miden las deposiciones energéticas de fotones, electrones y diferentes hadrones y el \acl{MS} permite detectar y medir las trayectorias de los muones. Entre ellos se encuentra un potente sistema magnético que curva la trayectoria de las partículas cargadas. Por último, el detector \ac{ATLAS} dispone de un preciso sistema de adquisición de datos y  trigger que filtra los eventos de poco interés para la física de altas energías, reduciendo así la frecuencia del flujo de datos en varios órdenes de magnitud para su almacenamiento. En los experimentos de física de altas energías es habitual trabajar con simulaciones tanto para la realización de medidas de precisión de los procesos conocidos del \ac{SM} como para comprender e interpretar las nuevas señales predichas por los escenarios \ac{BSM}. Esto añade otro grado de complejidad al experimento, ya que se requiere que estos datos sintéticos describan de forma excelente los procesos físicos que se obtienen de las colisiones del \ac{LHC}. 

Una parte central del programa de física de \ac{ATLAS} incluye la producción de fotones originados directamente en las colisiones protón-protón (\pp) en el \ac{LHC} -- llamados fotones prompt -- relevantes tanto para la realización de medidas precisas de observables que contribuyen a probar rigurosamente la \ac{QCD}, como para investigar escenarios \ac{BSM} que incluyen la generación de fotones prompt aislados en el estado final. Sin embargo, el principal proceso que tiene lugar en las colisiones \pp es la producción de jets los que, en algunas circunstancias como es el caso de la presencia en el jet de piones neutros muy energéticos, pueden tener características muy similares a las que tendría un fotón, por lo que este jet se puede identificar erróneamente como un fotón. El proceso de identificación de objetos físicos en \ac{ATLAS} constituye uno de los ingredientes principales en cualquier análisis de física. En el caso de los fotones, esta identificación se lleva a cabo estudiando la lluvia \ac{EM} iniciada por las partículas en el calorímetro utilizando diversas variables que describen su forma y, como se anticipó, se lleva a cabo utilizando tanto los datos como las simulaciones. Cabe mencionar que, debido al grado de precisión alcanzado en las medidas experimentales y a que las simulaciones no son perfectas, es imprescindible realizar correcciones a dichas simulaciones para que emulen los datos permitiendo su uso para la obtención de resultados coherentes en todas las investigaciones de \ac{ATLAS}. Una de las contribuciones de esta tesis consiste en la corrección de las distribuciones de las variables utilizadas para la identificación de fotones, introduciendo mejoras sustanciales en el método actual para corregir dichas distribuciones -- denominado \ac{FF} -- mejorando notablemente su performance. Asimismo, se introdujo otro enfoque en el que se realizan correcciones directamente en el nivel más bajo de las señales en los detectores, para así mejorar  simultáneamente el acuerdo dato-simulaciones en todas las variables construidas a más alto nivel que luego son utilizadas para identificar fotones.

Como se mencionó más arriba, los fotones prompt son de gran importancia para las búsquedas \ac{BSM}. En particular, en el estado final fotón+jet, la masa invariante de procesos del \ac{SM} sigue una forma descendente muy suave proporcionando un escenario excelente para las búsquedas de excesos localizados originados por nuevas resonancias de alta masa decayendo en  un par fotón+jet. Varios modelos teóricos que pretenden dar respuesta a las diferentes carencias del \ac{SM} predicen la existencia de este tipo de partículas. En particular, para dar una explicación de por qué existen tres familias de fermiones, se introdujeron modelos donde los quarks no son partículas fundamentales sino estados ligados de otras más fundamentales que experimentan una fuerza desconocida. Consecuentemente, podrían observarse estados de \ac{EQ}, \qstar,  en colisiones \pp en el \ac{LHC} dependiendo de si el valor de la escala de composición \(\Lambda\) es menor que la energía del centro de masa. Estos \ac{EQ} decaerían dando lugar a un par de fotón y jet en estado final, manifestándose como un \enquote{bump} en la distribución de masa invariante \myj alrededor de la masa del \ac{EQ}. Otras teorías, con la introducción de dimensiones extras, intentan encontrar una solución al problema de jerarquías. De particular importancia son los modelos de dimensiones extra que predicen que la escala fundamental de Planck \(m_P\) en las \(4 + n\) dimensiones (siendo \(n\) el número de dimensiones espaciales extra) está en la escala TeV y por tanto accesible a las energías del \ac{LHC}. En tal escala TeV, a partir de una masa umbral \(m_{\text{th}}\), se podrían producir \ac{QBH} en colisiones \pp del \ac{LHC} y luego decaer en un pequeño número de partículas de estado final incluyendo pares fotón-quark/gluón antes de que sean capaces de termalizarse. En este caso podría observarse una amplia estructura resonante justo por encima de \(m_{\text{th}}\) sobre la distribución de \myj del \ac{SM}. Bajo esta teoría se estudian dos modelos particulares que proponen diferentes números de dimensiones extra: el modelo RS1 de Randall-Sundrum propone un total de 5 dimensiones espacio-temporales, y el modelo ADD de Arkani-Hamed, Dimopoulos y Dvali que cuenta con un total de 10 dimensiones espacio-temporales. Por último, dada la suavidad de la distribución \myj, es posible realizar una búsqueda agnóstica de modelos sobre este fondo, en la que se considera que la señal resonante se evidencia como forma gaussiana. 

Estudios previos similares de búsquedas de resonancias en el estado final de \gammajet fueron realizados en el \ac{LHC}: el último resultado de \ac{ATLAS} utilizó \(36.7~\ifb\) y excluyó los modelos de \qstar con masas de hasta \(5.3~\TeV\), los \ac{QBH} del tipo RS1 hasta \(4.4~\TeV\) y los \ac{QBH} del tipo ADD hasta \(7.1~\TeV\). Por otro lado, \ac{CMS} también realizó estudios similares de modelos de \ac{EQ} con \(138~\ifb\), separando resonancias livianas (\qstar) y pesadas (\bstar), lo que resultó en límites de masas de \(6.0~\TeV\) y \(2.2~\TeV\), respectivamente. \ac{CMS} también estudió los modelos de \ac{QBH} ADD y RS1, extendiendo los límites superiores de las masas hasta \(7.5~\TeV\) y \(5.2~\TeV\), respectivamente.

El trabajo principal de esta tesis es la búsqueda de resonancias de alta masa en el estado final fotón+jet motivada por los modelos mencionados de \ac{EQ} y \ac{QBH}, con un análisis superador respecto de investigaciones previas tanto en \ac{ATLAS} como en \ac{CMS}. La búsqueda se realizó utilizando el conjunto completo de datos del Run-2 recolectados a una energía de centro de masa de \(\sqs = 13~\TeV\) resultando en un total de \(140.01~\ifb\). Para estudiar la elementalidad de los quarks, los resultados se interpretan con modelos de \ac{EQ}, investigando por primera vez en \ac{ATLAS} el efecto de diferentes acoplamientos con los campos de gauge del \ac{SM} (\(f\)) y separando en diferentes sabores, \qstar (\(u^*/d^*\)), \cstar y \bstar. Este trabajo de tesis incluye, asimismo, la primera búsqueda de quarks excitados en el canal fotón+jet en experimentos del \ac{LHC} considerando el sabor charm, quarks \(c\).

La tesis se divide en cuatro partes. La \Part{\ref{part:theory}} describe el marco teórico y las motivaciones del trabajo, presentando brevemente el \ac{SM} y los dos modelos teóricos que motivan la búsqueda de física \ac{BSM} de este trabajo. La \Part{\ref{part:exp_setup}} contiene dos capítulos en los que el primero (\Ch{\ref{ch:atlas}}) describe el \ac{LHC} y el detector \ac{ATLAS}, haciendo hincapié en las distintas partes del detector que son determinantes para este trabajo. En el \Ch{\ref{ch:objects}} se analizan los métodos utilizados para la reconstrucción e identificación de los distintos objetos físicos. En la \Part{\ref{part:pid}} se explican los diferentes algoritmos utilizados para la identificación de fotones. En el \Ch{\ref{ch:pid_ss}} se discuten las variables utilizadas para la identificación de fotones, cómo se optimiza y cómo se realizan las medidas de eficiencia de la identificación. En el \Ch{\ref{ch:ss_corrections}} se presentan los métodos desarrollados en esta tesis para corregir las distribuciones de las variables obtenidas con simulaciones que se utilizan para realizar la identificación de fotones. Finalmente, en la \Part{\ref{part:search}} se presenta la búsqueda de resonancias de \gammajet en el rango de altas masas. El \Ch{\ref{ch:strategy}} comienza discutiendo la estrategia general de análisis de datos y los métodos estadísticos que se aplicarán para obtener los resultados. En el \Ch{\ref{ch:samples}} se describen las muestras que se utilizan tanto para los modelos teóricos como para los fondos contaminantes de señal a partir de procesos del \ac{SM}. La selección de eventos y las definiciones de las regiones de señal que se utilizan en la búsqueda se presentan en el \Ch{\ref{ch:evt_selection}}. El modelado de la señal y las incertezas sistemáticas teóricas y experimentales se describen en el \Ch{\ref{ch:signals}}. En el \Ch{\ref{ch:bkg}} se estudian los fondos contaminantes a la señal búsqueda, siendo su determinación uno de los aspectos más complejos de la búsqueda de nueva física de este trabajo. En el \Ch{\ref{ch:results}} se presentan y discuten los resultados obtenidos en esta búsqueda, seguido por las conclusiones finales.