\chapter*{Introduction}
\addcontentsline{toc}{chapter}{Introduction}
\markboth{}{Introduction}



Our best current understanding of particle physics is given by the \ac{SM}, a theory that successfully explains a wide range of experimental results and precisely predicted many different physics phenomena and new particles, such as the Higgs boson discovered in 2012 by the \acs{ATLAS} and \ac{CMS} collaborations which lead to the Nobel Prize Award to François Englert and Peter Higgs the following year. Despite its remarkable success, the \ac{SM} is known to be incomplete as it cannot explain a number of experimental observations, such as the overwhelming astrophysical and cosmological evidence for dark matter, the hierarchy problem, why there are only three families of fermions, among others. In the past decades, many theories for new physics \ac{BSM}, such as \ac{SUSY}, emerged and provided well-motivated and promising theoretical frameworks to extend our fundamental understanding of particle physics and improve upon the shortcomings of the \ac{SM}. However, none of the numerous searches for new physics signatures that have been conducted at particle colliders in the past years could provide any direct evidence for the existence of new particles or forces as predicted by these theories.

A wide variety of these new theoretical models predict the existence of particles at high energies. In order to explore these regions, specially the \tev scale, in the \acused{CERN}\ac{CERN} laboratory the \ac{LHC}~\cite{LHC-Machine} was built. Installed in a 27-kilometre circular tunnel, it is the world's largest and most powerful particle collider. This machine is capable of colliding energetic beams of protons at rates upward of millions per second. The precision and high beam energy of the \ac{LHC} allow to explore energies above \(7~\tev\), an energy range never before achieved in a particle collider. The \ac{LHC} collides protons in four interaction points, where the 4 \ac{LHC} experiments are located: \acused{ATLAS}\ac{ATLAS}, \acused{CMS}\ac{CMS}, \acused{LHCb}\ac{LHCb} and \acused{ALICE}\ac{ALICE}.
Between the years 2015 and 2018 a data-taking period called Run-2 took place, where protons were collided at \(\sqs=13~\tev\), and collecting a total of integrated luminosity of \(140.01~\ifb\). In 2022, the \ac{LHC} Run-3 started, where the center-of-mass energy was increased to \(\sqs=13.6~\tev\), and by the end of 2024 \ac{ATLAS} could collect \(183~\ifb\) of data.

One of the most important experiments at the \ac{LHC} is \ac{ATLAS} (\acl{ATLAS}), a general-purpose detector designed to perform both precision measurements within the \ac{SM} and searches for new phenomena associated with physics \ac{BSM}. The \ac{ATLAS} detector is composed of different subdetectors that play different roles in the reconstruction of the colliding particles. The \acl{ID} is in charge of measuring the tracks of charged particles, the calorimeters measure the energetic depositions of photons, electrons and different hadrons, and finally the \ac{MS} allows to measure the muon trajectories. Intertwined between them there is a powerful magnet system, which bends the trajectory of charged particles. Finally, the \ac{ATLAS} detector has a precise Trigger System that filters out events of little interest, thus reducing the frequency of the data flow. 
In any high energy physics experiment, it is usual to work with simulations, both for the known processes of the \ac{SM}, to understand the shapes of new signals predicted by \ac{BSM} scenarios. This adds another degree of complexity to the experiment, as one requires the simulations to describe in an excellent way the real physics processes that one would get from the actual data.

Prompt photon production from \pp collisions at the \ac{LHC} constitues a key part on the \ac{ATLAS} physics program, either for precise measurements of \ac{QCD} observables, or beacause several \ac{BSM} scenarios involve having isolated prompt photons in the final state. However, the main process taing place in \pp collisions is dijet production, and sometimes one of these jets has a very similar signature as a photon would have, therefore this jet being mis-identified as a photon. The process of identification in \ac{ATLAS} constitutes one of the main ingredients in any physics analysis. For photons, this identification is carried out by studying the \ac{EM} shower initiated by the particles in the calorimeter using several shape variables, and as anticipated, it is carried out using the actual recollected data and the simulations. However, it was seen that the simulation did not correctly predict data leading to incongruent results.
One of the main tasks for this thesis is the correction of the shape variables used for photon identification. The current method to correct the variables is called \ac{FF}, which was drastically improved in this work. Furthermore, another approach was studied, where modifications to lower-level variables can be carried out to simultaneously fix all the variables used to identify photons.

It was mentioned in the previous paragraph that prompt photons are of great importance for \ac{BSM} searches. In particular, in the photon+jet final state, the invariant mass follows a very smoothly falling shape, providing an excellent scenario for bump-searches, where different particles that decay to a photon+jet pair can be searched for. Two of the theoretical models that aim to answer different \ac{SM}'s shortcomings predict the existence of these types of particles. The first one gives an explanation of why there are three fermion families, and propose that the quarks are not fundamental particles but bound states of more fundamental ones that experiment an unknown force. Then, \ac{EQ} states (\qstar) should be observed in \pp collisions at the \ac{LHC} depending if the value of compositeness scale \(\Lambda\) is lower than the center-of-mass energy. These \acp{EQ} would decay into a pair of photon and jet, leaving a bump on the \myj around the mass of the \ac{EQ}. The second model, with the introduction of extra dimensions, attempts to propose a solution for the hierarchy problem. Certain types of extra-dimension models predict the fundamental Planck scale \(m_P\) in the \(4 + n\) dimensions (\(n\) being the number of extra spatial dimensions) to be at the \tev scale, and thus accessible in \pp collisions at \(\sqs=13~\tev\) at the \ac{LHC}.
In such a \TeV-scale \(m_P\), \acp{QBH} may be produced at the \ac{LHC} as a continuum above the threshold mass \(m_{\text{th}}\) and then decay into a small number of final-state particles including photon-quark/gluon pairs before they are able to thermalize. In this case a broad resonance-like structure could be observed just above \(m_{\text{th}}\) on top of the \ac{SM} \myj distribution. Two particular models are studied under this theory, which propose different number of extra dimensions: the Randall-Sundrum RS1 model proposes a total of 5 space-time dimensions, and the Arkani-Hamed, Dimopoulos and Dvali ADD model, which counts with a total of 10 space-time dimensions.
Finally, given the smoothness of the \myj distribution, it is possible to make a model-agnostic search on this background, where the signal is considered to follow a Gaussian-shape resonance. This type of search provides a more general interpretation of the study, as it allows to compare with any theory model that proposes a Gauss-shape resonance.


Similar \gammajet resonance studies considering the same theoretical models have been carried out previously, setting upper limits on the theories. \ac{ATLAS} latest result used \(36.7~\ifb\) and excluded \qstar models with masses up to \(5.3~\tev\), RS1-type \ac{QBH} models up to \(4.4~\tev\) and ADD-type ones up to \(7.1~\tev\). On the other hand, \ac{CMS} also performed similar studies using \(138~\ifb\), where they studied \ac{EQ} models separating into light- (\qstar) and heavy-flavour (\bstar) \acp{EQ}, setting stringent limits of \(6.0~\tev\) and \(2.2~\tev\), respectively. \ac{CMS} also studied the ADD and RS1 \ac{QBH} models, where upper limits on the masses extend up to \(7.5~\tev\) and \(5.2~\tev\), respectively.

Then, the main work for this thesis is the search for high-mass resonances in the photon+jet final state. The search is performed using the full Run-2 dataset collected at a center-of-mass energy of \(\sqs=13~\tev\), using a total of \(140.01~\ifb\). Both the \ac{EQ} and \ac{QBH} models are studied, as well as the search for generic Gaussian-shaped signals.
In this work, \acp{EQ} models are studied separating into \qstar (\(u^*/d^*\)), \cstar and \bstar signals, being this work the first one at the \ac{LHC} considering the charm flavour, thanks to a novel and excellent-performing flavour tagger algorithm.




The thesis is divided into four parts. 
\Part{\ref{part:theory}} describes the theory background and motivations of the work, where \ac{SM} is briefly described and the two theory models for physics \ac{BSM} are discussed.

\Part{\ref{part:exp_setup}} contains two chapters in which the first one (\Ch{\ref{ch:atlas}}) describes the \ac{LHC} and the \ac{ATLAS} detector in detail, making emphasis on the different parts of the \ac{ATLAS} detector. In \Ch{\ref{ch:objects}}, the methods used for the different object reconstruction and identification are discussed.

In \Part{\ref{part:pid}}, the different algorithms used for photon identification are explained. First, in \Ch{\ref{ch:pid_ss}}, the variables used for photon identification, how it is optimised, and how the identification efficiency measurements are performed are discussed. Then, \Ch{\ref{ch:ss_corrections}} presents the different methods derived in this thesis to fix the photon shape variables in the simulation.

Finally, \Part{\ref{part:search}} presents the search for \gammajet resonances in the high-mass range. \Ch{\ref{ch:strategy}} starts discussing the general analysis strategy and sets up the statistical methods to be used. Next, in \Ch{\ref{ch:samples}}, the samples that are used for both the theory models as for the \ac{SM} backgrounds are described. The event selection and the definitions of the signal regions that are used in the search is presented in \Ch{\ref{ch:evt_selection}}. Signal modeling and the exerimental and theoretical systematic uncertainties are described in \Ch{\ref{ch:signals}}, which comprise one of the most important and challenging points in any search. It is crucial to have an excellent understanding of the \ac{SM} backgrounds. For this, in \Ch{\ref{ch:bkg}}, the two main backgrounds are studied, where a functional model is used to model the final, total, background. Finally, the results obtained from this search are presented and discussed in \Ch{\ref{ch:results}}.


%\epigraph{\emph{The journey, not the destination matters.}}{Thomas S. Eliot}


%Provide preliminary background information that puts your research in context (Why?)
%
%Clarify the focus of your study (What?)
%
%Point out the value of your research(including secondary research)!! (What gain?)
%
%Specify your specific research aims and objectives 

%The reader needs to know why your research is worth doing


%maybe want to motivate why looking at run-2 data? 

%At the time of writing
%between run 2 and run 3
%at time of writing,  run-3 data-taking is just about to start.. 
%important to use previous dataset to gauge limitations of searches
%--> crucial to contribute to the overall runnning of the experiment
%--> study and understand trigger 
%The result is striking. Two --> WIMP miracle!
%seemingly unrelated problems – the Higgs unnaturalness (rooted in the quantum structure
%of the particle world at distances of 10−20 meters and below) and the nature of dark matter
%(observed from galactic distances of 1020 meters to the largest scales in the universe)
%the science historian Thomas Kuhn
%We are confronted with the need to reconsider the guiding principles that have been used for decades to address the most fundamental questions about the physical world. These are symptoms of a phase of crisis.
%Greek krisis, which means “decisive moment”, “turning point”,
%privilege of the opportunity for an upcoming paradigm change.

% Within the last decades of particle physics research there have been many milestones and discoveries, narrowing in on the building blocks of matter and leading to the development of the Standard Model (\acs{SM}) of particle physics as we know it today. 
% This has been achieved through extraordinary efforts and interplay between experimental measurements unravelling hints and evidence of new particles,  and theoretical effort of tying this to an overarching theory and achieving more precise predictions:
% from the discovery of the electron in 1897 \cite{ElectronThomson},  to the discovery of the tau lepton in 1975 \cite{TauDiscovery},  to the most recent success of the theoretical predictions lying in the experimental discovery of the Higgs boson \cite{ATLASHiggsDiscovery,CMSHiggsDiscovery}.  In the last ten years since the Higgs boson's discovery,  its properties have been measured to high precision \cite{HiggsReviewATLAS10years} and the Standard Model’s predictions have been put to stringent tests. 
% In various measurements, tensions with predictions of the Standard Model have been appearing, hinting towards a larger underlying theory. The limitations of the Standard Model have become more and more evident. The Hierarchy problem,\cite{SUSYPrimer}, questioning the difference in scales between the Higgs mass and the Planck scale, raising the need for unnatural fine-tuning, is only one of the hints for the need for a larger principle. The evolution of our universe into a matter-dominated universe is so far unexplained, with no mechanisms within the Standard Model to sufficiently generate this asymmetry with anti-matter \cite{WMAP, Thomson,Sakharov}. Lastly, one of the most striking hints connects the limitations of the theoretical model describing the smallest building blocks in our universe to the largest structures known to humankind. The presence of Dark Matter \cite{Zwicky,RubinKent,RotationCurves,Planck,BulletCluster, OtherMergingClusters} in galaxies and galaxy clusters can not be explained through a composition of Standard Model particles. 

% An additional symmetry between bosons and fermions called \ac{SUSY} could offer solutions to many of these limitations. Not only could \ac{SUSY}  avoid the need for fine-tuning of the Higgs mass, but also \acp{WIMP} predicted by SUSY could make up a component of Dark Matter. This striking connection of two problems at length scales varying from the size of a fundamental particle to galaxies and galaxy clusters is known as the WIMP miracle. \\
% Supersymmetric particles have been searched for in many ways at the Large-Electron-Positron collider, the Large-Hadron-Collider as well as through a variety of non-collider experiments. Up to the moment of writing this thesis, there has been no direct detection of Dark Matter particles and no evidence of supersymmetric particles at colliders. 

% According to definitions of science historian Thomas Kuhn and discussion thereof of Gian Francesco Giudice (\cite{Kuhn,Giudice}) particle physics can currently be described to be in a period of "krisis".  Krisis has to be understood in its original meaning - a period of change and anticipation. This period can be frustrating and confusing, with a lack of direction to a new underlying principle,  but should be seen as a privilege.  A period preceding a paradigm change,  with room for creativity for new ideas but also the need for diligent exploration of limitations in current experiments.

% At the time of writing, a new data-taking period at the LHC has just begun. In preparation for this new data taking, it is crucial to thoroughly analyse the Run-2 data set, find its limitations and uncovered areas of new physics searches in order to prepare for the new challenges to come. 

% In this thesis, a search for \ac{SUSY} has been performed, looking for a production of the lightest chargino and next to lightest neutralino (supersymmetric partners of SM gauge bosons), decaying via a scalar tau lepton into a final state with hadronically decaying tau leptons. This search has been performed with data collected by the ATLAS detector at the LHC, as part of the ATLAS Collaboration. This final state with hadronically decaying tau leptons belongs to the ‘paths less walked’ within the ATLAS Collaboration, due to its challenging reconstruction. This offers an interesting window to determine and overcome the limitations of the ATLAS Collaboration’s search program for \ac{SUSY}. This analysis has been the author’s full responsibility. 
% Next to this main effort within this thesis, the performance of electron triggers within ATLAS have been studied as part of the author's qualification task as well as continued commitment to ensure the successful operation and good performance of ongoing ATLAS data-taking.

% The structure of this thesis is as follows: A brief overview of the theoretical concepts of the SM as well as its limitations,  motivating \ac{SUSY} is given in Chapter \ref{ch:theory}. This is followed by a conceptual description of LHC proton-proton collisions and the ATLAS detector in chapter \ref{ch:expsetup}. Further details on the data collection and reconstruction of collision events as well as the simulations used to study the events is given in Chapter \ref{ch:DAQ}. A detailed view on the electron trigger and its performance is given in Chapter \ref{ch:trigger}. The search for supersymmetric gauge bosons in all its necessary details is given in Chapter \ref{ch:analysis}. 