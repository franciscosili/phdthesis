\chapter*{Conclusions}
\addcontentsline{toc}{part}{Conclusions}
\markboth{}{Conclusions}
\epigraph{\emph{Don't let anyone rob you of your imagination, your creativity, or your curiosity.}} {Mae Jemison}

% In this thesis,  the extensive search program of the \ac{ATLAS} experiment for supersymmetric parti-cles was extended with a search for the production of the lightest chargino and next-to-lightest neutralino decaying via staus into a final state with at least two hadronically decaying tau leptons. 
% This thesis has extended the Run-2 \ac{ATLAS} \ac{SUSY} search program in these scenarios into a final state with at least two same-sign hadronic taus.  This final state had not been investigated during Run-2.
% The analysis has targeted not only higher \Cone / \Ntwo masses than excluded by a previous analysis with partial Run-2 data and opposite-sign taus,  but also aimed to push the sensitivity toward scenarios with small mass differences between the \Ntwo and \None and low \Cone masses. Dedicated signal regions have been designed for both scenarios. 

% A data-driven method was developed to estimate multi-jet backgrounds populating especially the low mass signal region.  Very good agreement between the background expectation in designated multi-jet backgrounds has been observed.  For top related background processes,  a control and validation region approach has been taken.  The choice of regions was designed to ensure a top background estimation including all type of tau contributions.  For \ac{SM} backgrounds  with a W boson in association with jets,  a dedicated strategy replacing one hadronically decaying tau with a light lepton has been designed.  Similarly for the validation of Multi-boson processes,  increased statistics in light-lepton final states have been utilized to estimate the background reliably.

% A variety of experimental and theoretical uncertainties have been evaluated and considered in this analysis,  ensuring a conservative but realistic statistical interpretation of the search results. 

% No significant excesses have been observed in the designed search regions.  The observed data events in both signal regions are within the \ac{SM} expectation.  In the absence of an excess,  this has been interpreted in 95\% confidence level exclusion limits in the \Cone, \None mass plane,  as well as an additional interpretation in the $\Delta m ($\Cone , \None$)$ vs \None mass plane.   

% A statistical combination with a search for the same final state topology in an opposite di-tau final state has been performed.  The combined model interpretation is increasing the excluded mass range up to \Cone masses of  1160 GeV for massless \None.  The same-sign analysis discussed in this thesis is significantly contributing to push the sensitivity towards low mass-splittings.  This region is challenging due to its soft decay products,  which is especially difficult in hadronic final states.  

% These efforts have further narrowed down on \ac{SUSY} and additionally provided valuable model-independent results in terms of upper cross section limits on \ac{BSM} signal contribution,  in an extreme di-tau kinematic phase space.  These model-independent results can be re-interpreted by the theory as well as extended experimental community. 

% Next to the search for new physics,  performance studies of the \ac{ATLAS} electron trigger have been presented.  These performance studies have provided crucial understanding of electron trigger behaviour throughout Run-2.  Studies on the electron trigger performance in AF2 detector simulation have highlighted the need to question and reassure previous modelling assumptions.

% With Run-3 of the \ac{LHC} just starting,  particle physics has an exciting period ahead.  The studies and analysis presented in this paper present a small but important wheel in the wide search program and \ac{ATLAS} operation to continue to learn and unravel the open questions in particle physics lying ahead. 



%This thesis presents a search for the production of supersymmetric gauge bosons decaying via scalar tau leptons into a final state of at least two hadronically decaying tau leptons.  This search has been performed analysing 13 TeV proton-proton collision data recorded by the ATLAS detector at the CERN Large-Hadron-Collider during the years of 2015-2018.  A total of 139 fb$^{-1}$ of data has been analyses.  No significant excesses have been observed with respect to the Standard Model expectation,  therefore exclusion limits have been set at 95\% confidence level.  Masses of the lightest chargino up to 950 GeV for massless lightest neutralinos have been excluded.  Mass differences between the \Cone and \Ntwo as little as 30 GeV have been excluded for a \Cone mass of 80 GeV.  A statistical combination with an ATLAS analysis using an opposite di-tau final state has been performed,  leading to an exclusion of \Cone masses up to 1150 GeV for massless \None~.  
%Dedicated studies on the performance of electron triggers are presented within this thesis that have been  part of the authors ATLAS qualification task.  These studies have hinted towards the need of dedicated trigger correction factors in AF2 fast simulation and has provided a first set of such scale factors to the ATLAS Collaboration. 