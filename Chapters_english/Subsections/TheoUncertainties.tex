%V+jets uncertainties

%7 point scale variations - muF mu R muF\&R variations
The experimental uncertainties described above affect the reconstructed objects and,  as a consequence,  the observed yields in data as well as simulation.  Additionally,  uncertainties connected to Monte Carlo simulation have to be taken into account.
These uncertainties can be estimated by varying the parameters in the Monte Carlo simulation,  affecting all aspect of Monte Carlo event generation. The evaluation of these uncertainties is described in the following for the main \ac{SM} backgrounds as well as \ac{SUSY} signal.  

\paragraph{Top related backgrounds}
Hard scattering and parton showering uncertainties have been extracted based on generator comparisons.  This has been evaluated separately for \ttbar\ as well as single top Wt contributions.  To avoid large uncertainties solely due to limited statistics available in the truth-level generator comparisons a loosened selection has been used.  
%tbar Nom (PhPy8)(826.251)
%ttbar (@MCNLO Py8)(761.982)
%ttbar (PhHerwig7)
The nominal generator used to simulate \ttbar\ events is a combination of \texttt{POWHEG} for the \ac{NLO} matrix element generation,  interfaced to \texttt{PYTHIA} for parton showering. 
A hard scattering,  matrix element related uncertainty is calculated through a comparison of truth-level Powheg Pythia 8 samples with \texttt{MadGraph5\_aMC@NLO} and \texttt{PYTHIA}.  
A variation in parton showering is achieved by replacing the \texttt{PYTHIA} shower through \texttt{Herwig7}.
The systematic uncertainty associated to these variations in regions $i$ is given by equation \ref{eq:theoUnc:var}.

\begin{equation}
\sigma^i = \frac{N_{nom}^i-N_{var}^i}{N_{nom}^i}
\label{eq:theoUnc:var}
\end{equation}
%\textcolor{red}{need to talk about LHE weight associated uncertainties as well as other weight-based uncertainties}
An additional weight has been introduced into the simulation to estimate the effect of a change in \ac{PDF} on the acceptance of the analysis.  An uncertainty related to the PDF variation has been calculated according to the PDF4LHC recommendations \cite{PDF4LHC}.  A similar weight has been included to estimate effects on \ac{ISR} and \ac{FSR} in for example showering or scale differences.  An uncertainty is extracted based on these internal reweighting of the Monte Carlo sample. 
An uncertainty accounting for individual variations in the factorisation and renormalisation scale is included.  Additionally,  a joint variation is taken into account.


\paragraph{Multi-boson backgrounds}
The \ac{PDF} as well as factorisation and renormalisation related uncertainties are estimated as described for the top backgrounds. 
Additionally for the Multi-boson backgrounds, uncertainties related to the matching between matrix element and parton showering as well as the resummation scale has to be taken into account.  Another part in the Monte Carlo event generation that can lead to different predicted event kinematics is the parton shower recoil scheme used. 
All of these possible variations in the Monte Carlo event generation are taken into account through an uncertainty,  which is extracted by comparing truth-level simulations with adapted Monte Carlo parameters.  Multi-boson processes are not normalised in control regions in the analysis described, therefore an additional uncertainty on the inclusive cross section of 20\% is included,  following recommendations of the ATLAS Multi-boson focus group.

\paragraph{V+jets backgrounds}
Vector boson production in association with jets is simulated using \textit{Sherpa} within this thesis.  Internal weight variations are used to assign an uncertainty accounting for the renormalisation, factorisation as well as PDF variations. 
For the considered \textit{Sherpa} samples,  a parametrised approach is used to estimate uncertainties related to the resummation and matching scales.  This parametrised approach is based on previous generator comparisons.
For $Z$ boson production with associated jets an additional uncertainty on the inclusive cross section of 5\% is included, following recommendation from the ATLAS Boson and Jets focus group.

\paragraph{Higgs related backgrounds}
Uncertainties on the modelling of Higgs boson related processes are included for scale variations as well as PDF variations.  Similar to top related processes, uncertainties connected to \ac{ISR} and \ac{FSR} are included.
Within the regions of the same-sign di-tau analysis,  Higgs related backgrounds are predominantly the production of a top pair in association with a Higgs. Following inclusive cross section uncertainty studies performed in \cite{CERNYellowReportHiggs},  an additional 15\% uncertainty is assigned to the Higgs background modelling. 


\paragraph{Signal cross section uncertainties}

For the simplified signal model,  similar modelling uncertainties due to the Monte Carlo generator configurations have to be taken into account.  One uncertainty is defined for a joint variation of the renormalisation and factorisation scale,  a second joint uncertainty considering the matching and merging scale.  Lastly,  a radiation uncertainty is taken into account. 
The uncertainties are extracted through comparisons to dedicated varied truth-level samples.  The uncertainty based on these variations have been evaluated for every mass point separately.  Due to limited available statistics in the truth samples,  loosened kinematic selections have been used.  To avoid further statistical fluctuations related to the limited statistics, an envelope of multiple averages over close-by mass points have been performed.  An uncertainty of
$ \pm 1.5\%$ will be considered as the joint renormalization and factorisation scale uncertainty. The merging scale variations are largely asymmetric, with its downward variation leading to very large uncertainties. To mititgate this, an averaged value of the upward fluctuation will be used, leading to a $\pm 7.0\%$ merging scale uncertainty.  A conservative 3.5\% will be used for the radiation uncertainty.

An overview of the theoretical uncertainty values is given in appendix \ref{app:theoUnc}.
