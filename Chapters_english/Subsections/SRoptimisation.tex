In the following, a description of the optimisation procedure used to define the signal regions is given.
A grid of simulated signal events obtained by varying the masses of \None and \Cone / \Ntwo has been used to perform the signal region optimisation.
The overall signal mass point grid can be seen in Figure \ref{fig:analysis:grid}.
%This includes an extension of the available mass points towards higher \Mntwo masses,  which where not yet available at the optimisation stage of the analysis described below.  This extension is not impacting the sensitivity reach or signal region definition of the analysis.

\subsection{Choice of kinematic variables}
To understand the difference between a potential supersymmetric particle production compared to a \ac{SM} process,  various kinematic variables can be studied.  \\
%This can be properties of the reconstructed particles, such as the transverse momentum ($p_T$),  pseudorapidity ($\eta$) and azimuthal angle $\Phi$,  as well as kinematic variables constructed from these properties.
One of the kinematic variables of interest in a di-tau final state is the \textit{visible mass} of the di-tau system, $m_{vis}(\tau_1,\tau_2)$, constructed using the four-momenta of the two leading tau candidates in the event.  This variable only takes into account the visible components of the hadronic decaying tau, therefore not considering the missing momentum carried by the tau neutrinos.

%In the case of a particle's decay product being invisible to the ATLAS detector,  a part of the momentum will be missing,  shifting the particle's invariant mass calculation.
To account for the invisible decay products,  the missing transverse momentum can be included in an adapted "invisible" mass, the transverse mass $m_T$:

\begin{align}
\label{eq:sr_optimisation:mt} m_T = \sqrt{2p_T^{\ell}E_T^{\text{miss}}(1-\cos(\Delta \phi(\vec{\ell},\vec{p}_T^{\text{miss}}))}
\end{align}

For example,  in the case of a leptonic W-bosons decay, the $m_T$ shows an endpoint at the mass of the W boson. \\
In the case of two hadronically decaying tau leptons,  the sum of the transverse masses $m_{Tsum}$ can be powerful in discriminating \ac{SM} background from signal:

\begin{align}
\label{eq:sr_optimisation:mtsum} m_{Tsum} = m_T(\tau_1)+m_T(\tau_2)
\end{align}

With $m_T(\tau_1)$ denoting the transverse mass as defined in equation \eqref{eq:sr_optimisation:mt} with the lepton as the leading,  highest momentum, hadronic decaying tau lepton. 
The transverse mass concept can be extended to pair produced particles partially decaying into invisible particles, using the definition by Barr, Lester,  Stephens and Summers \cite{mt2original, mt2discussion} of the stransverse mass:

\begin{align}
m_{T2} ( \textbf{p}_T^{\ell_1}, \textbf{p}_T^{\ell_2}, \textbf{p}_T^{\text{miss}}) \equiv  \underset{\textbf{p}_1^\text{miss} + \textbf{p}_2^\text{miss} = \textbf{p}_T^\text{miss}}{\text{min}} \left[ \text{max} \{ m_T(\textbf{p}_T^{\ell_1}, \textbf{p}_1^{\text{miss}}),   m_T(\textbf{p}_T^{\ell_2}, \textbf{p}_2^{\text{miss}}) \} \right]
\end{align}

Here the momentum ($\textbf{p}_T^{\ell_1}, \textbf{p}_T^{\ell_2}$) of the two leptons in the event (for example the visible part of the hadronic tau decay) are combined into a transverse mass with parts of the missing transverse momentum, split into $\textbf{p}_1^\text{miss}$ and $\textbf{p}_2^\text{miss}$.  
This variable shows an endpoint at the mass of the pair-produced particle,  leading to a usually lower end-point of the distribution for \ac{SM} background processes in comparison to the \ac{SUSY} model considered. The mass of the invisible particles is assumed to be zero.

If an event includes more than two leptons (with a maximum number of leptons $n_\ell$),  it can be useful to perform a maximisation over the $m_{T2}$ values resulting from all possible lepton pairs:

\begin{align}
m_{T2}^\text{max} = \max_{\substack{i = 1,...,n_\ell \\ j = 2,..., n_\ell \\ i \neq j}} \left[ m_{T2}(\textbf{p}_T^{\ell_i}, \textbf{p}_T^ {\ell_j},\textbf{p}_T^{\text{miss}}) \right] \label{eq:sr_optimisation:mt2max}
\end{align}

\FloatBarrier
\subsection{SR optimisation procedure}
As starting point of the optimisation,  the kinematic variables described in the previous section are studied for \ac{SM} background events and signal events. In the following plots, all \ac{SM} background distributions shown are obtained from Monte Carlo.
In Figure \ref{fig:app:LMpreselections}, some kinematic variables are shown after the selection with an asymmetric di-tau trigger selection as well as a b-jet veto and light lepton veto.  %These preselections are used to gauge the discriminating power of a set of kinematic variables for varying signal mass points.
A selection of signal points is used to gauge the effectiveness of the variables to isolate a signal.
In Figure \ref{fig:analysis:opt:mt2max},  $m_{T2}^\text{max}$,  as defined in \eqref{eq:sr_optimisation:mt2max} can be compared with Figure \ref{fig:analysis:opt:mt2},  showing $m_{T2}$. The maximisation over all tau pairs in the event leads to a slight improvement in sensitivity compared to $m_{T2}$ calculated using the two leading tau candidates.  In the lower panels of Figure \ref{fig:app:LMpreselections},  a simplified sensitivity measure $Z_n$ \cite{zndefinition} including a 30\% flat systematic uncertainty is used to gauge the sensitivity of each variable to the signal model.  A $Z_n$ implementation of the RooStats \cite{RooStats} package within the ROOT \cite{Root} software framework is used.  This sensitivity measure is used to estimate the signal to background separation power as fully estimated with hypothesis testing as described in section \ref{sec:analysis:stats}.
As can be seen in Figures \ref{fig:analysis:opt:mttau1} and Fig.  \ref{fig:analysis:opt:mttau2} in comparison to the sum of both $m_T$ shown in Figure \ref{fig:analysis:opt:summt},  summing the transverse mass of the two leading taus offers a slightly higher discriminating power than the single transverse masses.
%/mnt/lustre/projects/epp/general/atlas/dk352/C1N2_tau/plotting_kinematics/19v02plotting/April20/210420_optimising

\begin{figure}[htpb]
\centering
\subfloat[$m_{T2}^\text{max}$ \label{fig:analysis:opt:mt2max} ]{\includegraphics[width=0.49\textwidth]{figures/SignalOp_Thesis_260722_AG_noATLAS/asymbasicLLV_mt2maxSROpt.pdf}}
\subfloat[$m_{T2}$ \label{fig:analysis:opt:mt2}]{\includegraphics[width=0.49\textwidth]{figures/SignalOp_Thesis_260722_AG_noATLAS/asymbasicLLV_mt2SROpt.pdf}}\\
\subfloat[$m_T(\tau_1,E_T^\text{miss})$ \label{fig:analysis:opt:mttau1}]{\includegraphics[width=0.49\textwidth]{figures/SignalOp_Thesis_260722_AG_noATLAS/asymbasicLLV_Mt_Tau1metSROpt.pdf}}
\subfloat[$m_T(\tau_2,E_T^\text{miss})$ \label{fig:analysis:opt:mttau2}]{\includegraphics[width=0.49\textwidth]{figures/SignalOp_Thesis_260722_AG_noATLAS/asymbasicLLV_Mt_Tau2metSROpt.pdf}}\\
\subfloat[$m_{Tsum}$ \label{fig:analysis:opt:summt}]{\includegraphics[width=0.49\textwidth]{figures/SignalOp_Thesis_260722_AG_noATLAS/asymbasicLLV_SumMtTauSROpt.pdf}}
\subfloat[$m_{vis}(\tau_1,\tau_2)$ \label{fig:analysis:opt:mvis}]{\includegraphics[width=0.49\textwidth]{figures/SignalOp_Thesis_260722_AG_noATLAS/asymbasicLLV_mvisTauTauSROpt.pdf}}\\
\subfloat[$E_T^\text{miss}$ \label{fig:analysis:opt:met}]{\includegraphics[width=0.49\textwidth]{figures/SignalOp_Thesis_260722_AG_noATLAS/asymbasicLLV_metSROpt.pdf}}
%\subfloat[$meff$]{\includegraphics[width=0.45\textwidth]{figures/C1N2SS/SROptAppendix/asym_basicLLV_meff_mc16_a_e_d_.pdf}} \\
\subfloat[$\Delta\phi(\tau_1,\tau_2)$ \label{fig:analysis:opt:deltaphi}]{\includegraphics[width=0.49\textwidth]{figures/SignalOp_Thesis_260722_AG_noATLAS/asymbasicLLV_deltaPhiTauTauSROpt.pdf}}
%\subfloat[$\Delta\phi(\tau_1,E_T^\text{miss})$]{\includegraphics[width=0.45\textwidth]{figures/C1N2SS/SROptAppendix/asym_basicLLV_deltaPhiTau1Met_mc16_a_e_d_.pdf}}
\caption{Kinematic variable distributions for SS scenario after a loose preselection. Error bands include statistical uncertainties only.  The lower panel shows the significance $Z_n$ including a 30 \% systematic uncertainty assumption. 
\label{fig:app:LMpreselections}}
\end{figure}

%210420_improveSign

In case of a low m(\Cone,\Ntwo) (low mass scenario, SR-C1N2SS-LM),  a cut of $E_T^\text{miss} < 150 \text{ GeV}$ is introduced.  The case of high \Cone, \Ntwo mass (SR-C1N2SS-HM) is defined by $E_T^\text{miss}>~150 \text{ GeV}$. The effect of multiple cuts (not applied consecutively) on the overall sensitivity is given in Figure~\ref{fig:app:LMgridstudies}.



\begin{figure}[h]
\centering
\subfloat[$m_{T2}^\text{max} > 20 \text{ GeV} $]{\includegraphics[width=0.49\textwidth]{figures/C1N2SS/SROptAppendix_Improved/SignificancesTrueonlymt2_20.pdf}}
\subfloat[$m_{T2}^\text{max} > 50 \text{ GeV} $]{\includegraphics[width=0.49\textwidth]{figures/C1N2SS/SROptAppendix_Improved/SignificancesTrueonlymt2_50.pdf}} \\
\subfloat[$m_{Tsum} > 150 \text{ GeV}$]{\includegraphics[width=0.49\textwidth]{figures/C1N2SS/SROptAppendix_Improved/SignificancesTrueonlySumMt_150.pdf}}
\subfloat[$m_{Tsum} > 200 \text{ GeV}$]{\includegraphics[width=0.49\textwidth]{figures/C1N2SS/SROptAppendix_Improved/SignificancesTrueonlySumMt_200.pdf}}\\
\caption{ Significance estimation including a $30 \%$ flat systematic uncertainty.\label{fig:app:LMgridstudies}}
\end{figure}

An $m_{Tsum} > 200 \text{ GeV}$ selection was chosen because it gives a higher sensitivity toward the diagonal of the (\Cone/\Ntwo, \None) mass plane.
The additional requirement $\Delta\phi(\tau_1,\tau_2) > 1.5$ is included in the signal region definition,  to reduce the signal contamination when estimating the multi-jet background (see section \ref{sec:bkgestimation:ABCD}.
The optimisation for the high mass case followed a very similar approach, leading to the two orthogonal signal regions defined in Table~\ref{tab:SRdef}.
%This selection and an inversion of the transverse mass sum requirement offered the basis of the ABCD method,  allowing for a multi-jet estimation.  This is in detail described in section \ref{sec:bkgestimation:ABCD}.
%A preliminary estimation of the multi-jet background lead to the introduction of a $\Delta\phi(\tau_1,\tau_2)+\Delta\phi(\tau_1, E_T^\text{miss})>2.5$ cut to reduce the signal contamination within the ABCD regions.


%With a multi-jet background estimate included, multiple further cuts on kinematic variables have been tested according to their performance over the grid, with a particular focus on the sensitivitiy towards the kinematic diagonal (\Mnone close to \Mntwo). These include:
%\begin{itemize}
%\item $m_T(\tau_1)+m_T(\tau_2)$ > [240, 260, 280, 300,400,500]
%\item $m_{vis}(\tau_1,\tau_2) > 130$, $m_{vis}(\tau_1,\tau_2) > 400$, $m_{vis}(\tau_1,\tau_2) \in [130,400]$
%\item $E_T^\text{miss}$ > [50,60,70,80]
%\item $m_{T2}^\text{max}$ > [10,20,30,40,50,60,70,80]
%\item Number of signal jets < [2,3]
%\end{itemize}

%Multiple different combinations of these cuts have been taken into account and their effect considered singular as well as in combination. These iterative studies led to the definitions given in section \ref{sec:SR:SS}.

 %The sum of the leading taus transverse mass ($m_T(\tau_1)+m_T(\tau_2)$) was used to define ABCD regions and with that a signal region optimisation phase space, using a $m_T(\tau_1)+m_T(\tau_2) > 300 \text{GeV}$ requirement.

%%%%%%%%%%%%%%%%%%%%%%%%%%%%%%%%%%%%%%%%%%%%%%%%%%%%%%%%%%%%%%%%%%%%%%%%%%%%%%%%%%%


\begin{table}[htpb!]
\centering
\begin{tabular}{c|c} \hline
SR-C1N2SS-LM & SR-C1N2SS-HM \\ \hline \hline
\multicolumn{2}{c}{$>=$ 2 medium taus (SS) } \\
\multicolumn{2}{c}{$b$-jet veto} \\
$\Delta\Phi(\tau_1,\tau_2)  > 1.5 $&-\\
${N}_{jets} <$3 & - \\ \hline
$m_{Tsum}  > 200 \text{ GeV} $ & $m_{Tsum} > 450 \text{ GeV}$\\
\multicolumn{2}{c}{$ m_{T2}^{\text{max}} > 80 \text{ GeV}$  } \\\hline %\vspace{0.5cm}\\ \hline %\vspace{1cm}
 asymmetric di-tau trigger  & di-tau+\met trigger \\
$\met <$ 150 GeV    & $\met >$ 150 GeV          \\
\multicolumn{2}{c}{$\tau_{1}$ and $\tau_{2}$ $p_\text{T}$ requirements described in Table
  \ref{tab:analysis:trigger} in Section \ref{sec:analysis:eventselection}}\\
\hline
\end{tabular}
\caption{Summary of selection requirements for the signal regions.
\label{tab:SRdef} }
\end{table}

An overview of the event yields in both signal regions is shown in table \ref{tab:SS:SRyields}. All backgrounds apart from the Multi-jet background are taken from MC simulation with only statistical uncertainties included below.  The Multijet estimation is performed using a data-driven method,  discussed in detail in section \ref{sec:bkgestimation:ABCD}.

\begin{table}
\centering
% Preview source code for paragraph 0

\begin{tabular}{c|c|c}
\hline
SM-process & SR-C1N2SS-LM & SR-C1N2SS-HM\tabularnewline
\hline
\hline
%\hline
Top & $0.01\pm0.01$ & $0.84\pm0.36$\tabularnewline
%\hline
Multi-boson & $0.47\pm0.11$ & $0.81\pm0.21$\tabularnewline
%\hline
Multi-jet & $0.94\pm0.27$ & $ -0.086 \pm 0.31$\tabularnewline
%\hline
$W$+jets & $0.32\pm0.32$ & $0.10\pm0.10$\tabularnewline
%\hline
$Z$+jets & $0.20\pm0.20$ & $0.59\pm0.56$\tabularnewline
%\hline
Higgs & $0.00\pm0.00$ & $0.02\pm0.00$\tabularnewline
\hline
%\hline
SM total & $1.95 \pm 0.48$ & $2.35\pm0.80$\tabularnewline
%\hline
\hline
%Ref. point (325, 175) & $18.20\pm3.00$ & $2.91\pm0.51$\tabularnewline
Ref. point (325, 175) & $ 7.80 \pm 1.27 $& $ 2.26 \pm 0.71 $\tabularnewline
%\hline
%Ref. point (500, 300)  & $5.25\pm0.75$ & $4.68\pm10.78$\tabularnewline
Ref. point (500, 300)  & $ 3.78 \pm 0.65 $& $ 5.62 \pm 0.88 $\tabularnewline
%\hline
%Ref. point (900, 300) & $0.71\pm0.07$ & $6.37\pm0.21$\tabularnewline
Ref. point (900, 300) &  $ 0.84 \pm 0.07 $ & $ 6.23 \pm 0.21 $\tabularnewline
\hline
\end{tabular}
\caption{Number of events in the signal regions for SM backgrounds, including statistical
  uncertainties.  The SM MC backgrounds are normalised to 139~\ifb. The multi-jet contribution is estimated
  from data with a simplified ABCD method described in Section
  \ref{sec:bkgestimation:ABCD}. Only statistical uncertainty are considered for MC
  estimated processes and multi-jet.
%The proper results are showed in Table~\ref{ysr0} in Section \ref{sec:results-bkgfit}.
\label{tab:SS:SRyields}}
\end{table}

%%%%%%%%%%%%%%%%%%%%%%%%%%%%%%%%%%%%%%%%%%%%%%%%%%%%%%%%%%%%%%%%%%%%%%%%%%%%%%%%%%%
%%%%    N-1 plots
%%%%%%%%%%%%%%%%%%%%%%%%%%%%%%%%%%%%%%%%%%%%%%%%%%%%%%%%%%%%%%%%%%%%%%%%%%%%%%%%%%%

Kinematic distributions in the signal regions, with requirements on the shown variable removed ("N-1 distributions"), can be seen in figure \ref{fig:SS:SRlowmass} for SR-C1N2SS-LM,  and figure \ref{fig:SS:SRhighmass} for SR-C1N2SS-HM,  already including the multi-jet estimation as well as systematic uncertainties discussed in section \ref{sec:analysis:bkgestimation} and \ref{sec:analysis:systematics}, respectively. 

\begin{figure}[!htpb]
\centering
  \subfloat[$\Delta\Phi(\tau_1,\tau_2)$]{\includegraphics[width=0.49\textwidth]{figures/UpdatedThesisFigures_noATLAS/LowMass/DeltaPhi_N1_improved.pdf}}
  \subfloat[${N}_{jets} $]{\includegraphics[width=0.49\textwidth]{figures/UpdatedThesisFigures_noATLAS/LowMass/SigJets_N1_improved.pdf}}\\
  \subfloat[$ m_{T2}^{\text{max}} $]{\includegraphics[width=0.49\textwidth]{figures/UpdatedThesisFigures_noATLAS/LowMass/Mt2_N1_improved.pdf}}
  \caption{``N-1'' distributions of relevant kinematic variables after SR-C1N2SS-LM requirements, except the one on the
shown variable, have been applied.
The stacked histograms show the expected SM backgrounds estimated from \ac{MC}, normalised to 139~\ifb as well as Multi-jet expectation as described in section \ref{sec:bkgestimation:ABCD}.
\label{fig:SS:SRlowmass}Statistical uncertainties and systematic uncertainties as discussed in section \ref{sec:analysis:systematics} are included in the error band. }
\end{figure}

\begin{figure}[!htpb]
\centering
\subfloat[$m_{Tsum}$]{\includegraphics[width=0.49\textwidth]{figures/UpdatedThesisFigures_noATLAS/HighMass/SumMt_N1_improved.pdf}}
\subfloat[$ m_{T2}^{\text{max}} $]{\includegraphics[width=0.49\textwidth]{figures/UpdatedThesisFigures_noATLAS/HighMass/Mt2_N1_improved.pdf}}
  \caption{``N-1'' distributions of relevant kinematic variables after SR-C1N2SS-HM requirements, except the one on the
shown variable, have been applied. The stacked histograms show the expected \ac{SM} backgrounds estimated from \ac{MC}, normalised to 139 ~\ifb as well as Multi-jet expectation as described in section \ref{sec:bkgestimation:ABCD}. Statistical and systematic uncertainties as discussed in \ref{sec:analysis:systematics} are included in the error band. 
\label{fig:SS:SRhighmass} }
\end{figure}

Further kinematic distribution in both signal regions can be seen in \ref{fig:SS:SRlowDistr} and \ref{fig:SS:SRhighDistr}, respectively. This clearly shows that no further requirements on other kinematic variables are able to gain in sensitivity,  at least not while allowing for sufficient remaining statistics in the \ac{SM} background simulations.

\begin{figure}[!htpb]
\centering
\subfloat[]{\includegraphics[width=0.49\textwidth]{figures/UpdatedThesisFigures_noATLAS/LowMass/asymSRD_wDPhiOnly_TRIALsummt200_mvis_nomvisup_MT280_deltaphi_signaljets2_noLLV_nomvis_deltaPhiTau1MetSRL.pdf}}
\subfloat[]{\includegraphics[width=0.49\textwidth]{figures/UpdatedThesisFigures_noATLAS/LowMass/asymSRD_wDPhiOnly_TRIALsummt200_mvis_nomvisup_MT280_deltaphi_signaljets2_noLLV_nomvis_deltaR_etaSRL.pdf}}\\
\subfloat[]{\includegraphics[width=0.49\textwidth]{figures/UpdatedThesisFigures_noATLAS/LowMass/asymSRD_wDPhiOnly_TRIALsummt200_mvis_nomvisup_MT280_deltaphi_signaljets2_noLLV_nomvis_metSRL.pdf}}
\subfloat[]{\includegraphics[width=0.49\textwidth]{figures/UpdatedThesisFigures_noATLAS/LowMass/asymSRD_wDPhiOnly_TRIALsummt200_mvis_nomvisup_MT280_deltaphi_signaljets2_noLLV_nomvis_tau1PtSRL.pdf}}\\
\subfloat[]{\includegraphics[width=0.49\textwidth]{figures/UpdatedThesisFigures_noATLAS/LowMass/asymSRD_wDPhiOnly_TRIALsummt200_mvis_nomvisup_MT280_deltaphi_signaljets2_noLLV_nomvis_tau2PtSRH.pdf}}
\subfloat[]{\includegraphics[width=0.49\textwidth]{figures/UpdatedThesisFigures_noATLAS/LowMass/asymSRD_wDPhiOnly_TRIALsummt200_mvis_nomvisup_MT280_deltaphi_signaljets2_noLLV_nomvis_mvisTauTauSRL.pdf}}
  \caption{Remaining kinematic distributions in SR-C1N2SS-LM, for all variables not used in the signal region definition.  The error band includes statistical and systematic uncertainties. The lower panel shows the significance $Z_n$ including a 30 \% systematic uncertainty assumption. 
\label{fig:SS:SRlowDistr}}.
\end{figure}

\begin{figure}[!htpb]
\centering
  \subfloat[]{\includegraphics[width=0.49\textwidth]{figures//UpdatedThesisFigures_noATLAS/HighMass/ditauMETSRSummthighmt2_deltaEtaTauTau.pdf} }
  \subfloat[]{\includegraphics[width=0.49\textwidth]{figures//UpdatedThesisFigures_noATLAS/HighMass/ditauMETSRSummthighmt2_deltaPhiTau1MetSRL.pdf} }\\
  \subfloat[]{\includegraphics[width=0.49\textwidth]{figures//UpdatedThesisFigures_noATLAS/HighMass/ditauMETSRSummthighmt2_deltaPhiTauTauSRH.pdf} }
     \subfloat[]{\includegraphics[width=0.49\textwidth]{figures//UpdatedThesisFigures_noATLAS/HighMass/ditauMETSRSummthighmt2_deltaR_etaSRL.pdf} }\\
  \subfloat[]{\includegraphics[width=0.49\textwidth]{figures/UpdatedThesisFigures_noATLAS/HighMass/ditauMETSRSummthighmt2_metSRH.pdf} }
  \subfloat[]{\includegraphics[width=0.49\textwidth]{figures/UpdatedThesisFigures_noATLAS/HighMass/ditauMETSRSummthighmt2_mvisTauTauSRH.pdf} }
  \caption{Remaining kinematic distributions in SR-C1N2SS-HM, for all variables not used in the signal region definition. The error band includes statistical and systematic uncertainties. The lower panel shows the significance $Z_n$ including a 30 \% systematic uncertainty assumption. 
\label{fig:SS:SRhighDistr}}
\end{figure}


%%%%%%%%%%%%%%%%%%%%%%%%%%%%%%%%%%%%%%%%%%%%%%%%%%%%%%%%%%%%%%%%%%%%%%%%%%%%%%%%%%%
%%%% sensitivity
%%%%%%%%%%%%%%%%%%%%%%%%%%%%%%%%%%%%%%%%%%%%%%%%%%%%%%%%%%%%%%%%%%%%%%%%%%%%%%%%%%%
 The estimated sensitivity of both signal regions, using the significance measure $Z_n$ is shown in figure \ref{fig:SS:SRsign}. This is a first estimate of the sensitivity of these regions to the gaugino pair production under investigation.

\begin{figure}[!htpb]
\centering
  \subfloat[SR-C1N2SS-LM]{\includegraphics[width=0.49\textwidth]{figures/C1N2SS/SRdefs/LowMass/SRLowMassSign_improved.pdf}}
  \subfloat[SR-C1N2SS-HM]{\includegraphics[width=0.49\textwidth]{figures/C1N2SS/SRdefs/HighMass/SRHighMassSign_improved.pdf}}
  \caption{Significance distribution for the (a) SR-C1N2SS-LM, (b)
  SR-C1N2SS-HM for 139~\ifb. The statistical uncertainty
 and 30\% systematic uncertainty on SM backgrounds are included in the $\text{Z}_{\text{N}}$
calculation. 
\label{fig:SS:SRsign}}
\end{figure}
