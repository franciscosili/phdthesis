% % Abstract

\thispagestyle{empty}
\pdfbookmark[0]{Resumen}{Resumen} % Bookmark name visible in a PDF viewer

% \begin{center}
%     %	\bigskip
%     {\normalsize \myUni \\} % University name in capitals
%     {\normalsize \myFaculty \\} % Faculty name
%     {\normalsize \myDepartment \\} % Department name
%     \bigskip\vspace*{.02\textheight}
%     {\Large \textsc{Tesis de Doctorado}}\par
%     \bigskip

%     {\rule{\linewidth}{1pt}\\%[0.4cm]
%     \Large \myTitle \par \mySubtitle} % Thesis title
%     \rule{\linewidth}{1pt}\\[0.4cm]

%     \bigskip
% 	{\normalsize por \myName \par} % Author name
%     \bigskip\vspace*{.06\textheight}
% \end{center}


{\centering\Huge\textsc{\textbf{Resumen}} \par}
\bigskip


En esta tesis se presentan investigaciones originales en física de altas energías con especial énfasis en la búsqueda de nueva física utilizando datos recolectados por el experimento \acs{ATLAS} a partir de colisiones protón-protón del \ac{LHC} a energía de centro de masa de \(\sqs = 13~\TeV\). Estas búsquedas, focalizadas en procesos con estados finales con un fotón y un jet, se llevaron a cabo a partir del desarrollo de estrategias novedosas para la identificación de resonancias provenientes de partículas o de estados de alta masa no incluidos en el \ac{SM}, que se manifiestan como un exceso localizado de eventos sobre el fondo del \ac{SM}. Los resultados obtenidos permitieron establecer los límites más estrictos en la sección eficaz visible de producción de resonancias independiente de modelos, así como de resonancias predichas en dos modelos de física más allá del \ac{SM}. Por un lado, para estudiar la elementalidad de los quarks, se consideraron modelos de Quarks Excitados con diferentes acoplamientos (\(f\)) y diferentes sabores, \qstar (\(u^*/d^*\)), \cstar y \bstar, en el que para el caso de \(f = 1\) se excluyeron masas de hasta \(6162\), \(3396\) y \(2470~\GeV\), respectivamente. Cabe destacar que los límites en el modelo de \cstar  corresponden a los primeros resultados obtenidos en estos procesos en los experimentos del \ac{LHC}. Por otro lado, se estudiaron modelos de Micro-Agujeros Negros, como los propuestos por Randall-Sundrum y Arkani-Hamed-Dimopoulos-Dvali que consideran la existencia de una y seis dimensiones extras, respectivamente, para los que los resultados obtenidos permitieron excluir masas de hasta \(5355~\GeV\) y \(7583~\GeV\), respectivamente. A su vez, se realizó un estudio sobre las correcciones a las distribuciones de las variables que describen el pasaje de los fotones por los calorímetros electromagnético y hadrónico -- imprescindibles para la correcta identificación de fotones en el detector \acused{ATLAS}\ac{ATLAS} -- que resultó en una mejora sustancial al método tradicional y se consideró, también, un nuevo enfoque en el que las correcciones se realizan en el nivel más bajo de la reconstrucción de señales en el detector. El desarrollo de este último método resulta de gran interés para la colaboración \ac{ATLAS} en vistas a la utilización de algoritmos de Machine Learning para la identificación de fotones, en particular en el contexto del futuro acelerador High-Luminosity \ac{LHC} (HL-LHC).


\noindent 
