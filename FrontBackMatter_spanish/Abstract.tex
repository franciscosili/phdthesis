% % Abstract

\thispagestyle{empty}
\pdfbookmark[0]{Resumen}{Resumen} % Bookmark name visible in a PDF viewer

% \begin{center}
%     %	\bigskip
%     {\normalsize \myUni \\} % University name in capitals
%     {\normalsize \myFaculty \\} % Faculty name
%     {\normalsize \myDepartment \\} % Department name
%     \bigskip\vspace*{.02\textheight}
%     {\Large \textsc{Tesis de Doctorado}}\par
%     \bigskip

%     {\rule{\linewidth}{1pt}\\%[0.4cm]
%     \Large \myTitle \par \mySubtitle} % Thesis title
%     \rule{\linewidth}{1pt}\\[0.4cm]

%     \bigskip
% 	{\normalsize por \myName \par} % Author name
%     \bigskip\vspace*{.06\textheight}
% \end{center}


{\centering\Huge\textsc{\textbf{Resumen}} \par}
\bigskip



Esta tesis presenta una búsqueda de nueva física en eventos que contienen un fotón y un jet de alto momento transverso, realizada con datos recolectados por el experimento \acs{ATLAS} a partir de colisiones protón-protón del \ac{LHC}, a energía de centro de masa de \(\sqs=13~\tev\).
La búsqueda consiste en la identificación de resonancias en la masa invariante del fotón y el jet - provenientes de partículas o de estados de alta masa no incluídos en el \ac{SM} - que se manifiestan como un exceso localizado de eventos sobre el fondo del \ac{SM}.
% Dado que no se observó un exceso estadísticamente significativo, se establecieron límites a \(95\%\) \acs{CL} en la sección eficaz de producción de resonancias Gaussianas genéricas.
Los resultados obtenidos permitieron establecer los límites más estrictos en la sección eficaz de producción de resonancias gaussianas genéricas y de aquellas predichas en dos modelos de física más allá del \ac{SM}.
Por un lado, un modelo de Quarks Excitados con diferentes acoplamientos (\(f\)) y diferentes sabores, \qstar (\(u^*/d^*\)), \cstar y \bstar, en el que para el caso de \(f=1\) se excluyeron masas de hasta \(6174\), \(3414\) y \(2493~\gev\), respectivamente.
Los límites en el modelo de \cstar corresponden a los primeros resultados obtenidos en experimentos del \acs{LHC}.
Por otro lado, se estudiaron modelos de Micro-Agujeros Negros, como los propuestos por Randall-Sundrum (Arkani-Hamed-Dimopoulos-Dvali) que consideran la existencia de una (seis) dimension(es) extra(s). En este caso, los resultados obtenidos permitieron excluir masas de hasta \(5347~\gev\) (\(7590~\gev\)).
A su vez, se realizó un estudio sobre las correcciones a las variables que describen el pasaje de los fotones por los calorímetros electromagnético y hadrónico, las cuales son imprescindibles para la correcta identificación de fotones en el detector \acs{ATLAS} y que resultó en una mejora sustancial al método tradicional. Se presentó, asimismo, un nuevo enfoque en el que las correcciones se realizan en el nivel más bajo de la reconstrucción de señales en el detector, siendo este de gran interés para la colaboración en vistas al futuro High-Luminosity \ac{LHC} y la utilización de algoritmos de Machine Learning.


\noindent 
